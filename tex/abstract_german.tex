\section*{Zusammenfassung}
\addcontentsline{toc}{section}{Zusammenfassung}


% Differenzielle Produktionswirkungsquerschnitte des Higgs-Bosons sind 
% empfindliche Proben f{\"u}r die Physik jenseits des Standardmodells.
% 
Der differenzielle Wirkungsquerschnitt des Higgs-Bosons ist ein 
empfindlicher Indikator f{\"u}r Physik jenseits des Standardmodells.
% 
% Der erste Teil dieser Arbeit befasst sich mit der Messung des 
% Higgs-Boson-Transversalimpulses, der Higgs-Boson-Rapidit{\"a}t, des 
% Transversalimpulses des f{\"u}hrendes Teilchenjet, und der 
% Jet-Multiplizit{\"a}tsspektren, wobei als Eingaben Messungen von den 
% Zerfallskan{\"a}len zu zwei Photonen, vier Leptonen und ein 
% Bottom-Antibottom-Quark-Paar in einer 'boosted' Topologie wirden benutzt.
% 
% Der erste Teil dieser Doktorarbeit befasst sich mit der Messung des 
% transversalen Impulses (oder Transversalimpulses) des Higgs-Bosons, der 
% Rapidit{\"a}t des Higgs-Bosons, des Transversalimpulses des f{\"u}hrenden 
% Teilchenjets und des Jet-Multiplizit{\"a}tsspektrum.
% 
Der erste Teil dieser Doktorarbeit befasst sich mit der Messung der Spektra des 
transversalen Impulses des Higgs-Bosons, der 
Rapidit{\"a}t des Higgs-Bosons, des Transversalimpulses des f{\"u}hrenden 
Teilchenjets und der Jet-Multiplizit{\"a}t.
% 
Hierzu werden die 
Messungen der Zerfallskan{\"a}le des Higgs-Bosons zu zwei Photonen ($\hgg$), zu vier 
Leptonen ($\hzz$) und zu einem Bottom-Antibottom-Quark-Paar in einer `boosted' 
Topologie ($\hbb$) herangezogen. % (oder Anordnung) 
% 
% Die Eingaben verwenden Proton-Proton-Kollisionsdaten, die mit dem 
% CMS-Detektor bei s = 13 aufgezeichnet wurden, was einer integrierten 
% Luminosit{\"a}t von 35,9 fb entspricht.
% 
Diese Messungen verwenden Daten an Proton-Proton-Kollisionen, die mit 
dem CMS Detektor bei $\sqrt{s}=13$\TeV aufgezeichnet wurden, mit einer 
integrierten Luminosit{\"a}t von 35.9\fbinv.
% 
% Die Genauigkeit des kombinierten Transversalimpulsspektrums wird um etwa 
% 15\% in Bezug auf den $\hgg$ kanal individuell verbessert.
% 
Die Pr{\"a}zision des Transversalimpulsspektrums wird durch diese 
Kombination um etwa 15\% bez{\"u}glich des einzelnen $\hgg$ Kanals verbessert.
% 
% Eine Kombination des Totale Wirkungsquerschnitts basierend auf dem 
% Diphoton und den vier Lepton-Zerfallskan{\"a}len wird ebenfalls durchgef{\"u}hrt 
% und wird mit 61 gemessen.
% 
Der totale Wirkungsquerschnitt wurde zu $61.1   \pm 6.0 \,\text{(stat)} \pm 3.7 \,\text{(syst)}$\pb gemessen, basierend auf der Kombination der $\hgg$ und $\hzz$ Zerfallskan{\"a}len.
% auf der Kombination der Diphoton und vier-Leptonen Zerfallskan{\"a}len.


% Der zweite Teil dieser Arbeit befasst sich mit der Interpretation des 
% Higgs-Boson-Transversalimpulsspektrums in Bezug auf 
% Higgs-Boson-Kopplungsmodifikatoren, die f{\"u}r zwei theoretische Modelle, 
% basiert auf effektiver Feldtheorie, durchgef{\"u}hrt wird.
% 
Der zweite Teil dieser Arbeit befasst sich mit der Interpretation des 
Higgs-Boson-Transversalimpulsspektrums bez{\"u}glich 
Higgs-Boson-Kopplungsmodifikatoren. Zwei theoretische Modelle der 
effektiven Feldtheorie werden dazu herangezogen.
% 
% Die erste betrifft ein Modell mit der Kopplung an das Top-Quark, die 
% Kopplung an das Bottom-Quark und eine anomale direkte Kopplung an das 
% Gluonenfeld als Parameter.
% 
Das erste Modell beinhaltet eine Kopplung an das Top-Quark, das 
Bottom-Quark und eine anomale direkte Kopplung an das Gluonfeld. % <-- anormale??
% 
% Dieses Modell erzeugt Abweichungen vom Standardmodell im Bereich des 
% großen Transversalimpulses und reagiert empfindlich auf endliche 
% Quarkmasseneffekte und neue Physik in der Gluon-Gluon-Fusionsschleife.
% 
Dieses Modell erzeugt Abweichungen vom Standardmodell f{\"u}r grosse 
Transversalimpulse und reagiert empfindlich auf finite % endliche?
Quarkmasseneffekte und neue Physik in der Gluon-Gluon-Fusionsschleife.
% 
% Die zweite betrifft ein Modell mit den Kopplungen nach Bottom- und 
% Charm-Quarks als Parametern, das auf Abweichungen des weichen (soft?) 
% Teils des Transversalimpulsspektrums anspricht.
% 
Die zweite Interpretation betrifft ein Modell mit Kopplung an Bottom- 
und Charm-Quark als Parameter, das den niederenergetischen Teil des 
Transversalimpulsspektrums adressiert.
% 
% Das Transversalimpulsspektrum und die Interpretation in Bezug auf die 
% Higgs-Boson-Kopplungen wird auf eine integrierte Luminosit{\"a}t von 3000 
% ifb projiziert, was die erwartete, integrierte Gesamt-Luminosit{\"a}t des 
% High-Luminosity LHC ist.
% 
Das Transversalimpulsspektrum und die Interpretation bez{\"u}glich der 
Higgs-Boson-Kopplungen wird auf eine integrierte Luminosit{\"a}t von
$3000\fbinv$ projiziert, was die erwartete totale integrierte Luminosit{\"a}t des 
High-Luminosity LHC ist.



% Der dritte und letzte Teil dieser Arbeit behandelt die Bestimmung der 
% starken Kopplungskonstante unter Verwendung von 
% Top-Antitop-Quarkpaarwirkungquerschnittsmessungen.
% 
Der dritte und letzte Teil dieser Arbeit behandelt die Bestimmung der 
starken Kopplungskonstante $\alphas$ unter Verwendung von 
Wirkungsquerschnittsmessungen von Top-Antitop-Quark-Paaren.
% 
% Das Ermittlungsverfahren folgt einem ersten das von der 
% CMS-Kollaboration durchgef{\"u}hrt wurde, wobei sich die Behandlung von 
% Skalenunsicherheiten und die Wahl der Partonverteilungsfunktionsgruppen 
% deutlich unterscheiden.
% 
Die Bestimmung folgt dem ersten von der CMS-Kollaboration 
durchgef{\"u}hrten Verfahren, wobei sich die Behandlung von 
Skalenunsicherheiten und die Wahl der Partondichtefunktionen deutlich 
unterscheiden.
% 
% Die Eingaben sind jeweils drei Messungen aus den CMS- und 
% ATLAS-Kollaborationen bei s = 7, 8 und 13 tev und eine kombinierte 
% Messung aus den CDF- und D0-Kollaborationen bei s = 1.
% 
Die benutzten Daten sind jeweils drei Messungen aus den CMS- und 
ATLAS-Kollaborationen bei $\sqrt{s}=7$, $8$, und $13\TeV$ und eine kombinierte 
Messung aus den CDF- und D0-Kollaborationen bei $\sqrt{s}=1.96\TeV$.
% 
% Unter Verwendung eines wahrscheinlichkeitsbasierten Ansatzes werden die 
% einzelnen Bestimmungen zu einer einzigen bestm{\"o}glichen Sch{\"a}tzung 
% zusammengefasst, wobei die Korrelationsstruktur der Unsicherheiten 
% sorgf{\"a}ltig ber{\"u}cksichtigt wird.
% 
Unter Verwendung eines wahrscheinlichkeitsbasierten Ansatzes werden 
die einzelnen Messungen zu einer einzigen bestm{\"o}glichen Sch{\"a}tzung 
zusammengefasst, wobei die Korrelationsstruktur der Messungenauigkeiten 
sorgf{\"a}ltig ber{\"u}cksichtigt wird.
% 
% Das Endergebnis ist a = 1.
% 
Das Endergebnis ist $\asmz = \alphasResultCenter^{+\alphasResultRightError}_{-\alphasResultLeftError}$.
