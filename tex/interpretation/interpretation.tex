\section{Interpretation of differential Higgs boson production cross sections}
\label{sec:interpretation}

\emph{%
This section first covers the possible interpretations of differential Higgs boson production cross sections.
%
The $\pth$ spectrum obtained in Section~\ref{sec:diffxs-results}, the most sensitive to potential deviations from the SM, is then interpreted in terms of Higgs boson coupling modifiers.
% 
For the interpretation of the $\pth$ spectrum, the text, illustrations, results, and conclusions closely follow the original documentation, which can be found in Refs.~\cite{AN-17-244,HIG-17-028}.
}


The central theme in the interpretation of differential cross sections is the use of the \emph{shape} of the differential distribution in addition to the overall normalization (which is obtained from inclusive measurements).
% 
The shape is in this thesis defined as the differential cross section normalized by the inclusive cross section:
% 
\begin{linenomath*}
\begin{equation}
\vec{s} = \frac{\vec{\Delta\sigma}}{\sigma_\text{incl}}
\,.
\end{equation}
\end{linenomath*}
% 
The shape of a differential distribution is a vector of measurable physical observables, which can be compared to theoretical predictions to obtain constraints on underlying physical parameters, or to search for deviations of the SM in order find (signs of) new physics.
% 
In this thesis most attention is given to $\pth$ spectrum, for the reason that its shape is expected to be the most sensitive to certain physical parameters of interest, such as Higgs boson couplings.



% ____________________________________________________________________________
\subimport{./}{theories}
\subimport{./}{systematics}
\subimport{./}{results}