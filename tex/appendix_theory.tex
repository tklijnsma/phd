\clearpage
\section{Spontaneous symmetry breaking in an Abelian field theory}
\label{app:theory}

This section shows how the expansion around the vacuum expectation and a clever gauge choice for a complex scalar field $\phi$ leads to spontaneous symmetry breaking in an Abelian field theory.
% 
The considered Lagrangian, repeated from Equation~(\ref{eq:lagrangian-higgs}), is
% 
\begin{linenomath*}
\begin{equation}
\label{eq:lagrangian-higgs-copy}
\lagrangian =
    -\frac{1}{4} F_{\mu\nu} F^{\mu\nu}
    + (\partial^\mu - \imag e A^\mu) \phi^\ast (\partial_\mu + \imag e A_\mu) \phi
    - \underbrace{(\mu^2 \phi^\ast \phi + \lambda (\phi^\ast \phi)^2)}_{V(\phi)}
\,,
\end{equation}
\end{linenomath*}
% 
and the transformation to be applied to the Lagrangian is
% 
\begin{linenomath*}
\begin{equation}
\label{eq:transformation-higgs-copy}
\phi \to \exp{i\alpha(x)} \frac{1}{\sqrt{2}} \left( v + \phi_1(x) \right)
\quad
A_\mu(x) \to A_\mu(x) - \frac{1}{e} \partial_\mu \alpha(x)
\,.
\end{equation}
\end{linenomath*}
% 
Since we will need them soon, it also worth explicitly writing down the transformation of $\partial_\mu \phi$:
% 
\begin{linenomath*}
\begin{equation}
\partial_\mu \phi \to
    \frac{1}{\sqrt{2}} \exp{\imag \alpha(x)} \left(
    \partial_\mu \phi_1 + \imag (v+\phi_1) \partial_\mu \alpha(x)
    \right)
\quad
\partial^\mu \phi^\ast \to
    \frac{1}{\sqrt{2}} \exp{-\imag \alpha(x)} \left(
    \partial^\mu \phi_1 - \imag (v+\phi_1) \partial^\mu \alpha(x)
    \right)
\,.
\end{equation}
\end{linenomath*}
% 
It is trivial to show that the first term, $\frac{1}{4} F_{\mu\nu} F^{\mu\nu}$, is invariant under the transformation under consideration.
% 
The covariant derivative is a little trickier; expanding the term in the Lagrangian yields
% 
\begin{linenomath*}
\begin{equation}
\label{eq:covariant-derivative}
(\partial^\mu - \imag e A^\mu) \phi^\ast (\partial_\mu + \imag e A_\mu) \phi
=
\underbrace{\partial^\mu \phi^\ast \partial_\mu \phi}_1
- \underbrace{\imag e A^\mu \phi^\ast \partial_\mu \phi}_2
+ \underbrace{\partial^\mu \phi^\ast \imag e A_\mu \phi}_3
+ \underbrace{e^2 A^\mu A_\mu}_4
\,.
\end{equation}
\end{linenomath*}
% 
It turns out none of the terms $1$ to $4$ are invariant by themselves, but combined the non-invariant contributions cancel beautifully.
% 
Transforming term $1$ yields:
% 
\begin{linenomath*}
\begin{equation}
\partial^\mu \phi^\ast \partial_\mu \phi \to
    \frac{1}{2} \partial^\mu \phi_1 \partial_\mu \phi_1
    + \frac{1}{2} (v+\phi_1)^2 \partial_\mu \alpha(x) \partial^\mu \alpha(x)
\,.
\end{equation}
\end{linenomath*}
% 
Terms $2$ and $3$ contain many invariance-violating contributions:
% 
\begin{linenomath*}
\begin{equation}
\begin{split}
-\imag e A^\mu \phi^\ast \partial_\mu \phi
% 
& \to -\imag e \frac{1}{2}
    (A^\mu - \frac{1}{e} \partial^\mu \alpha(x) )
    (v+\phi_1) \exp{-\imag \alpha(x)}
    \exp{\imag \alpha(x)} \left(
        \partial_\mu \phi_1 + \imag (v+\phi_1) \partial_\mu \alpha(x)
        \right)
    \\
% 
& \;= -\imag e \frac{1}{2} (v+\phi_1) \left[
    A^\mu \partial_\mu \phi_1
    - \frac{1}{e} \partial^\mu \alpha(x) \partial_\mu \phi_1
    + \imag A^\mu (v+\phi_1) \partial_\mu \alpha(x)
    - \imag \frac{1}{e} \partial^\mu \alpha(x) (v+\phi_1) \partial_\mu \alpha(x)
    \right]
    \\
% 
& \;= -\frac{1}{2} (v+\phi_1) \left[
    \imag e A^\mu \partial_\mu \phi_1
    - i \partial^\mu \alpha(x) \partial_\mu \phi_1
    - e A^\mu (v+\phi_1) \partial_\mu \alpha(x)
    + \partial^\mu \alpha(x) (v+\phi_1) \partial_\mu \alpha(x)
    \right]
% 
\,,
\end{split}
\end{equation}
\end{linenomath*}
% 
\begin{linenomath*}
\begin{equation}
\begin{split}
\imag e \partial^\mu \phi^\ast A_\mu \phi
% 
& \to \imag e \frac{1}{2}
    \exp{-\imag \alpha(x)} \left(
        \partial^\mu \phi_1 - \imag (v+\phi_1) \partial^\mu \alpha(x)
        \right)
    (A_\mu - \frac{1}{e} \partial_\mu \alpha(x) )
    (v+\phi_1) \exp{\imag \alpha(x)}
    \\
% 
& \;= \imag e \frac{1}{2} (v+\phi_1) \left[
    \partial^\mu \phi_1 A_\mu 
    - \imag (v+\phi_1) \partial^\mu \alpha(x) A_\mu
    - \frac{1}{e} \partial^\mu \phi_1 \partial_\mu \alpha(x)
    + \imag \frac{1}{e} (v+\phi_1) \partial^\mu \alpha(x) \partial_\mu \alpha(x)
    \right]
    \\
% 
& \;= \frac{1}{2} (v+\phi_1) \left[
    \imag e \partial^\mu \phi_1 A_\mu 
    - \imag \partial^\mu \phi_1 \partial_\mu \alpha(x)
    + e (v+\phi_1) \partial^\mu \alpha(x) A_\mu
    - (v+\phi_1) \partial^\mu \alpha(x) \partial_\mu \alpha(x)
    \right]
% 
\,,
\end{split}
\end{equation}
\end{linenomath*}
% 
but their sum already cancels most of them:
% 
\begin{linenomath*}
\begin{equation}
\begin{split}
- \imag e A^\mu \phi^\ast \partial_\mu \phi
+ \imag e \partial^\mu \phi^\ast A_\mu \phi
% 
&\to -(v+\phi_1) \left[
    - e A^\mu (v+\phi_1) \partial_\mu \alpha(x)
    + \partial^\mu \alpha(x) (v+\phi_1) \partial_\mu \alpha(x)
    \right]
    \\
% 
&\;=
    e (v+\phi_1)^2 \partial_\mu \alpha(x) A^\mu
    - (v+\phi_1)^2 \partial_\mu \alpha(x) \partial^\mu \alpha(x)
\end{split}
\end{equation}
\end{linenomath*}
% 
And finally, term $4$:
% 
\begin{linenomath*}
\begin{equation}
\begin{split}
e^2 A^\mu \phi^\ast A_\mu \phi
% 
&\to \frac{1}{2} e^2 (v+\phi_1)^2
    (A^\mu - \frac{1}{e} \partial^\mu \alpha(x) )
    (A_\mu - \frac{1}{e} \partial_\mu \alpha(x) )
    \\
% 
&\;= \frac{1}{2} e^2 (v+\phi_1)^2 \left[
    A^\mu A_\mu
    - 2 \frac{1}{e} A^\mu \partial_\mu \alpha(x)
    + \frac{1}{e^2} \partial^\mu \alpha(x) \partial_\mu \alpha(x)
    \right]
    \\
% 
&\;= 
    \frac{1}{2} e^2 (v+\phi_1)^2 A^\mu A_\mu
    - e (v+\phi_1)^2 A^\mu \partial_\mu \alpha(x)
    + \frac{1}{2} (v+\phi_1)^2 \partial^\mu \alpha(x) \partial_\mu \alpha(x)
\,.
\end{split}
\end{equation}
\end{linenomath*}
% 
Inserting the individual transformed terms back into Equation~(\ref{eq:covariant-derivative}), we notice all the invariance-violating contributions cancel, and we obtain
% 
\begin{linenomath*}
\begin{equation}
(\partial^\mu - \imag e A^\mu) \phi^\ast (\partial_\mu + \imag e A_\mu) \phi
\to
    \frac{1}{2} \partial^\mu \phi_1 \partial_\mu \phi_1
    + \frac{1}{2} e^2 (v+\phi_1)^2 A^\mu A_\mu
\,.
\end{equation}
\end{linenomath*}
% 
Next we consider the transformed potential.
% 
It pays of to first transform $\phi^\ast\phi$ and $(\phi^\ast\phi)^2$:
% 
\begin{linenomath*}
\begin{equation}
\phi^\ast\phi
\to 
\frac{1}{2} (v+\phi_1^2) = \frac{1}{2} v^2 + v\phi_1 + \frac{1}{2} \phi_1^2
\,,
\end{equation}
\end{linenomath*}
% 
\begin{linenomath*}
\begin{equation}
(\phi^\ast\phi)^2
\to 
\frac{1}{4}v^4 + v^3\phi_1 + \frac{3}{2} v^2 \phi_1^2 + v\phi_1^3 + \frac{1}{4}\phi_1^4
\,.
\end{equation}
\end{linenomath*}
% 
The potential then follows trivially:
% 
\begin{linenomath*}
\begin{equation}
V(\phi) = \mu^2 \phi^\ast \phi + \lambda (\phi^\ast \phi)^2
\to
\frac{1}{4}\lambda\phi_1^4 + v\lambda\phi_1^3 - \mu^2\phi_1^2 + \frac{1}{4}\mu^2 v^2
\,.
\end{equation}
\end{linenomath*}
% 
Inserting the transformed covariant derivative and potential into the Lagrangian then yields the transformed Lagrangian:
% 
\begin{linenomath*}
\begin{equation}
\lagrangian \to \lagrangian^\prime =
    -\frac{1}{4} F_{\mu\nu} F^{\mu\nu}
    + \frac{1}{2} \partial^\mu \phi_1 \partial_\mu \phi_1
    + \frac{1}{2} e^2 (v+\phi_1)^2 A^\mu A_\mu
    - \frac{1}{4}\lambda\phi_1^4 - v\lambda\phi_1^3 + \mu^2\phi_1^2 - \frac{1}{4}\mu^2 v^2
\,.
\end{equation}
\end{linenomath*}
% 
As described in the main text, the fields that appear here can be interpreted as physical, massive fields.
% 
The original symmetry, described by the transformations in Equation~(\ref{eq:transformation-higgs-copy}), is no longer valid for the appearing physical fields, hence the symmetry is `broken'.





