\section*{Abstract}
\addcontentsline{toc}{section}{Abstract}

Differential production cross sections of the Higgs boson are sensitive probes for physics beyond the standard model.
% 
The first part of this thesis concerns the measurement of the Higgs boson transverse momentum, the Higgs boson rapidity, the transverse momentum of the leading jet, and the jet multiplicity spectra, using as input measurements from the decay channels to two photons ($\hgg$), four leptons ($\hzztofourl$), and a bottom-antibottom quark pair in a boosted topology ($\hbb$).
% 
The inputs use proton-proton collision data recorded with the CMS detector at $\sqrt{s}=13$\TeV corresponding to an integrated luminosity of 35.9\fbinv.
% 
The precision on the combined transverse momentum spectrum is improved by about 15\% with respect to the $\hgg$ channel individually.
% 
A combination of the inclusive cross section based on the $\hgg$ and $\hzztofourl$ decay channels is also performed, and is measured to be $61.1   \pm 6.0 \,\text{(stat)} \pm 3.7 \,\text{(syst)}$\pb.


The second part of this thesis concerns the interpretation of the Higgs boson transverse momentum spectrum in terms of Higgs boson coupling modifiers, which is carried out for two theoretical models based on effective field theory.
% 
The first concerns a model with the coupling to the top quark, the coupling to the bottom quark, and an anomalous direct coupling to the gluon field as parameters.
% 
This model produces deviations from the standard model in the large transverse momentum regime, and is sensitive to finite quark mass effects and new physics in the gluon-gluon fusion loop.
% 
The second concerns a model with the couplings to the bottom and charm quarks as parameters, which is sensitive to deviations of the soft part of the transverse momentum spectrum.
% 
The transverse momentum spectrum and the interpretation in terms of the Higgs boson couplings is projected to an integrated luminosity of $3000\fbinv$, which is the expected total integrated luminosity of the High-Luminosity LHC.


The third and final part of this thesis treats the determination of the strong coupling constant $\alphas$ using inclusive top-antitop quark pair production cross section measurements.
% 
The determination procedure follows one first carried out by the CMS Collaboration, with notable differences in the treatment of scale uncertainties and the choice of the parton distribution function sets.
% 
The inputs are three measurements from the CMS and ATLAS Collaborations each, at $\sqrt{s}=7$, $8$, and $13\TeV$, and one combined measurement from the CDF and D0 Collaborations at $\sqrt{s}=1.96\TeV$.
% 
Using a likelihood-based approach, the individual determinations are combined into a single best estimate, taking carefully into account the correlation structure of the uncertainties.
% 
The final results is $\asmz = \alphasResultCenter^{+\alphasResultRightError}_{-\alphasResultLeftError}$.



% Differential Higgs boson ($\hboson$) production cross sections are sensitive probes for physics beyond the standard model.
% % 
% New physics may contribute in the gluon-gluon fusion loop, the dominant Higgs boson production mechanism at the LHC, and manifest itself through deviations from the expected standard model distributions.
% % 
% % Combined spectra from the $\PH\to\cPgg\cPgg$, $\PH\to\cPZ\cPZ$ and $\PH\to\bbbar$ decay channels are presented, together with limits on the Higgs boson couplings and a combined inclusive Higgs boson production cross section measurement using proton-proton collision data recorded with the CMS detector at $\sqrt{s}=13$\TeV, corresponding to an integrated luminosity of 35.9\fbinv.
% % 
% Spectra for the $\hgg$, $\hzztofourl$, and $\hbb$ decay channels, using proton-proton collision data recorded with the CMS detector at $\sqrt{s}=13$\TeV corresponding to an integrated luminosity of 35.9\fbinv, are presented.
% % 
% The spectra are combined to place limits on the Higgs boson couplings to the top, bottom, and charm quarks, as well as a direct coupling to the gluon field, and to measure the inclusive Higgs boson production cross section.
% % 
% The measured total cross section is $61.1   \pm 6.0 \,\text{(stat)}   \pm 3.7 \,\text{(syst)} \,$pb, and the precision of the measurement of the differential cross section of the Higgs boson transverse momentum is improved by about 15\%.


\section*{Zusammenfassung}
\addcontentsline{toc}{section}{Zusammenfassung}

(Abstract in German)
