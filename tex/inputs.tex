\section{Inputs to the combined differential cross section measurement and interpretation}
\label{sec:inputs}

\emph{%
This section covers object reconstruction at CMS in general, and the analysis strategies for the input analyses of the combination of differential cross sections.
% 
}

\tk{
For each of the decay channels:
\begin{itemize}
\item Objects reconstruction and identification
\item Event selection and categorization
\item Analysis strategy
\end{itemize}
}
% ____________________________________________________________________________



% ____________________________________________________________________________
\subsection{\texorpdfstring{$\hgg$}{H to gamma gamma}}

Because of the excellent photon energy resolution of the CMS detector, the $\hgg$ decay channel is a clean decay channel with an excellent mass resolution.
% 
Its signature is a reasonably sharp peak in the diphoton mass spectrum around 125\GeV, on top of an irreducible QCD background.
% 
With its good signal-to-background ratio and good energy resolution, it was the driving input for the discovery of the Higgs boson in 2012~\cite{Aad:2012tfa,Chatrchyan:2012xdj,Chatrchyan:2013lba}, despite its relatively small branching fraction.



% In the case of the $\hgg$ analysis, the fiducial phase space is defined by requiring the ratio of the leading (subleading) photon $\pt$ to the diphoton mass to be greater than $1/3$ ($1/4$).
% % 
% In addition, for each photon candidate the scalar sum of the generator-level $\pt$ of stable particles contained in a cone of radius $\Delta R=0.3$ around the candidate is required to be less than 10\GeV, where $\Delta R = \sqrt{\smash[b]{(\Delta\eta)^2+(\Delta\phi)^2}}$ is the angular separation between particles and $\Delta\phi$ is the azimuthal angle between two particles in radians.
% % 
% The selected photon pairs are categorized according to their estimated relative invariant mass resolution~\cite{Sirunyan:2018kta}.

\subsection{\texorpdfstring{$\hzztofourl$}{H to ZZ to four leptons}}

\tk{TODO}

\subsection{\texorpdfstring{$\hbb$}{H to bb}}

\tk{TODO}


\subsection{Paper text}

For all the analyses used as input to the combination ($\hgg$~\cite{Sirunyan:2018kta}, $\hzztofourl$~\cite{CMS_AN_2016-442}, and $\hbb$~\cite{CMS_AN_2016-366}), the data set corresponds to an integrated luminosity of $35.9\fbinv$ recorded by the CMS experiment in 2016.
The $\hbb$ decay channel is only included in the combination of the $\pth$ spectra, improving the measurements at the higher end of the distribution ($\pth > 350$\GeV) where the data from the $\hgg$ and $\hzz$ decay channels are limited.
All analyses provide the parametrization of the folding matrix $M_{ji}^{k}$ (which is the probability for an event in generator-level bin $i$ to be reconstructed in bin $j$ and category $k$) in terms of a common generator-level binning, that is used for the combined spectra.
Given the limited statistical precision in the individual channels, the results of the $\hzz$ and $\hbb$ channels individually are reported for a coarser binning, which is provided in Tables~\ref{tab:binningpth}--\ref{tab:binningptjet} for each of the observables.
This binning coincides with the binning at the reconstruction level.


The SM prediction for the differential cross sections is simulated with $\MGvATNLO$ v2.2.2~\cite{Alwall:2014hca} for each of the four dominant Higgs boson production modes: gluon-gluon fusion (\ggh), vector boson fusion, associated production with a $\wboson$/$\zboson$ boson, and associated production with a top quark-antiquark pair.
A contribution from Higgs boson production in association with bottom quarks is not simulated, but included assuming its acceptance is equal to that from Higgs boson production via gluon fusion.
The matrix element calculation includes the emission of up to two additional partons and is performed at NLO accuracy in perturbative quantum chromodynamics (QCD).
Events are interfaced to \PYTHIA8.205~\cite{Sjostrand:2014zea} for parton showering and hadronization with the CUETP8M1~\cite{Skands:1695787} underlying event tune.
The matrix element calculation is matched to the parton shower following the prescription in Ref.~\cite{Frederix:2012ps}.
A weight depending on $\pth$ and $\njets$ is applied to simulated $\ggh$ events to match the predictions from the {\textsc{nnlops}} program~\cite{Hamilton:2012np, Kardos:2014dua}, as discussed in Ref.~\cite{Sirunyan:2018koj}.
The set of parton distribution functions used in all simulations is NNPDF3.0~\cite{Ball:2014uwa}.
The hadronic jets are clustered from the particle-flow candidates~\cite{Sirunyan:2017ulk} in the case of data and simulation, and from stable particles excluding neutrinos in the case of generated events, using the anti-$\kt$ clustering algorithm~\cite{Cacciari:2008gp} with a distance parameter of $0.4$.
The measurements are reported in terms of kinematic observables defined before the decay of the Higgs boson, i.e. at the generator level.


Each of the analyses used as input to the combination corresponds to a different fiducial phase space definition and applies a different event categorization.
In the case of the $\hgg$ analysis, the fiducial phase space is defined by requiring the ratio of the leading (subleading) photon $\pt$ to the diphoton mass to be greater than $1/3$ ($1/4$).
In addition, for each photon candidate the scalar sum of the generator-level $\pt$ of stable particles contained in a cone of radius $\Delta R=0.3$ around the candidate is required to be less than 10\GeV, where $\Delta R = \sqrt{\smash[b]{(\Delta\eta)^2+(\Delta\phi)^2}}$ is the angular separation between particles and $\Delta\phi$ is the azimuthal angle between two particles in radians.
The selected photon pairs are categorized according to their estimated relative invariant mass resolution~\cite{Sirunyan:2018kta}.
In the case of the $\hzz$ analysis, the 4-lepton mass is required to be greater than 70\GeV, the leading $\zboson$ boson candidate invariant mass must be greater than 40\GeV, and leptons must be separated in angular space by at least $\Delta R > 0.02$.
Furthermore, at least two leptons must each have a $\pt>10$\GeV and at least one a $\pt > 20$\GeV.
The selected events are categorized according to their lepton configuration in the final state (4 electrons, 4 muons, or 2 electrons and 2 muons).
In the case of the $\hbb$ analysis, the analysis strategy requires the presence of a single anti-$\kt$ jet with a distance parameter of $0.8$, $\pt>450\,$GeV, and $\abs{\eta}<2.5$.
For this analysis, the data is not unfolded to a fiducial phase space.
Soft and wide-angle radiation is removed using the soft-drop grooming algorithm~\cite{Dasgupta:2013ihk,Larkoski:2014wba}.
The jet mass after application of the soft-drop algorithm, $\msd$, peaks close to the Higgs boson mass in the case of signal events.
To avoid finite-cone effects and the nonperturbative regime of the $\msd$ calculation, events are selected based on the dimensionless mass scale variable for QCD jets defined as $\rho=\log\left(\msd^2/\pt^2\right)$~\cite{Dasgupta:2013ihk}, which relates the jet $\pt$ to the jet mass.
Events with isolated electrons, muons, or \taulepton leptons with $\pt>10$\GeV and $\abs{\eta}<2.5$ are vetoed in order to reduce the background from SM electroweak processes, and events with a missing transverse momentum greater than $140$\GeV are vetoed in order to reduce the background from top quark-antiquark pair production.
Additionally, a selection criterion is applied based on the compatibility of the single anti-$\kt$ jet with having a two-prong substructure~\cite{Dolen:2016kst,Moult:2016cvt,Larkoski:2013eya,Thaler:2010tr}.
Events are categorized according to their likelihood of consisting of two $\bquark$ quarks, which is computed using the double-$\bquark$ tagger algorithm~\cite{Sirunyan:2017ezt}.


Minor modifications are applied to the individual analyses in Refs.~\cite{Sirunyan:2018kta,CMS_AN_2016-442,CMS_AN_2016-366} to provide the inputs used for the combination of differential observables.
For $\hgg$, an additional bin, $\pth>600$\GeV, is included in the $\pth$ spectrum.
For $\hzz$, the binning is modified for multiple kinematic observables to align with the binning of the $\hgg$ analysis.
Furthermore, the branching fractions of the two $\zboson$ bosons to the various lepton configurations are fixed to their SM values, whereas in Ref.~\cite{CMS_AN_2016-442} these are allowed to float.
For $\hbb$ the signal is split into two $\pt$ bins at the generator level:
the first with $350\le\pt<600$\GeV and the second an overflow bin with $\pt\ge600$\GeV, which aligns with the binning of the other channels.
At the reconstruction level two bins are employed, with $400\le\pt<600$ and $\pt\ge600$\GeV, which is a slight modification with respect to the binning used in Ref.~\cite{CMS_AN_2016-366}.
The redefinition of the reconstructed $\pt$ categories necessitates a reevaluation of the background model, which is performed using the same procedure as in the original analysis.
For the purpose of the combination in this analysis, the fiducial measurements from the $\hgg$ and $\hzz$ channels are extrapolated to the inclusive phase space~\cite{Alwall:2014hca,Hamilton:2012np,Kardos:2014dua}.


\begin{table*}[htb]
    \centering
    \topcaption{
        The reconstruction-level binning for $\pth$ for the $\hgg$, $\hzz$, and $\hbb$ channels.
        This binning coincides with the binning of the unfolded cross sections in which the individual results are reported.
        }
    \label{tab:binningpth}
    \tabletextwidth{
    \setlength{\tabcolsep}{5pt}
    \begin{tabular}{lccccccccc}
    Channel & \multicolumn{9}{l}{$\pth$ binning (GeV)} \\[\tablelineskip]
    \hline
    $\hgg$
        & [0, 15)    & [15, 30)   & [30, 45)   & [45, 80)        & [80, 120)
        & [120, 200) & [200, 350) & [350, 600) & [600, $\infty$)
        \\
    $\hzz$
        & [0, 15) & [15, 30)
        & \multicolumn{2}{l}{[30,  80)}
        & \multicolumn{2}{l}{[80,  200)}
        & \multicolumn{3}{l}{[200, $\infty$)}
        \\
    $\hbb$
        & \multicolumn{7}{@{{}}c@{{}}}{None} & [350, 600) & [600, $\infty$)
        \\
    \end{tabular}
    }
    \end{table*}

\begin{table}[htb]
    \centering
    \topcaption{
        The binning for $\njets$ for the $\hgg$ and the $\hzz$ channels.
        This binning coincides with the binning of the unfolded cross sections in which the individual results are reported.
        }
    \label{tab:binningnjets}
    \begin{tabular}{lccccc}
    Channel & \multicolumn{5}{l}{$\njets$ binning} \\[\tablelineskip]
    \hline
    $\hgg$ & 0 & 1 & 2 & 3 & $\ge$4 \\
    $\hzz$ & 0 & 1 & 2 & \multicolumn{2}{l}{$\ge$3} \\
    \end{tabular}
    \end{table}

\begin{table*}[htb]
    \centering
    \topcaption{
        The binning for $\absy$ for the $\hgg$ and the $\hzz$ channels.
        This binning coincides with the binning of the unfolded cross sections in which the individual results are reported.
        }
    \label{tab:binningabsy}
    \begin{tabular}{lcccccc}
    Channel & \multicolumn{6}{l}{$\absy$ binning} \\[\tablelineskip]
    \hline
    $\hgg$ & [0.0, 0.15) & [0.15, 0.30) & [0.30, 0.60) & [0.60, 0.90) & [0.90, 1.20) & [1.20, 2.50] \\
    $\hzz$ & [0.0, 0.15) & [0.15, 0.30) & [0.30, 0.60) & [0.60, 0.90) & [0.90, 1.20) & [1.20, 2.50] \\
    \end{tabular}
    \end{table*}

\begin{table*}[h!]
    \centering
    \topcaption{
        The binning for $\ptjet$ for the $\hgg$ and the $\hzz$ channels.
        This binning coincides with the binning of the unfolded cross sections in which the individual results are reported.
        }
    \label{tab:binningptjet}
    \begin{tabular}{lcccccc}
    Channel & \multicolumn{6}{l}{$\ptjet$ binning (GeV)} \\[\tablelineskip]
    \hline
    $\hgg$ & [0, 30) & [30, 55) & [55, 95) & [95, 120) & [120, 200) & [200, $\infty$) \\
    $\hzz$ & [0, 30) & [30, 55) & [55, 95) & \multicolumn{3}{l}{ [95, $\infty$) } \\
    \end{tabular}
    \end{table*}

