\section{Inputs to the combined differential cross section measurement and interpretation}
\label{sec:inputs}

\emph{%
This section covers object reconstruction at CMS in general, and the analysis strategies for the input analyses of the combination of differential cross sections.
% 
}

% ____________________________________________________________________________
\subsection{Event simulation}

\tk{Paper text:} The SM prediction for the differential cross sections is simulated with $\MGvATNLO$ v2.2.2~\cite{Alwall:2014hca} for each of the four dominant Higgs boson production modes: gluon-gluon fusion (\ggh), vector boson fusion, associated production with a $\wboson$/$\zboson$ boson, and associated production with a top quark-antiquark pair.
% 
A contribution from Higgs boson production in association with bottom quarks is not simulated, but included assuming its acceptance is equal to that from Higgs boson production via gluon fusion.
% 
The matrix element calculation includes the emission of up to two additional partons and is performed at NLO accuracy in perturbative quantum chromodynamics (QCD).
% 
Events are interfaced to \PYTHIA8.205~\cite{Sjostrand:2014zea} for parton showering and hadronization with the CUETP8M1~\cite{Skands:1695787} underlying event tune.
% 
The matrix element calculation is matched to the parton shower following the prescription in Ref.~\cite{Frederix:2012ps}.
% 
A weight depending on $\pth$ and $\njets$ is applied to simulated $\ggh$ events to match the predictions from the {\textsc{nnlops}} program~\cite{Hamilton:2012np, Kardos:2014dua}, as discussed in Ref.~\cite{Sirunyan:2018koj}.
% 
The set of parton distribution functions used in all simulations is NNPDF3.0~\cite{Ball:2014uwa}.
%
The hadronic jets are clustered from the particle-flow candidates~\cite{Sirunyan:2017ulk} in the case of data and simulation, and from stable particles excluding neutrinos in the case of generated events, using the anti-$\kt$ clustering algorithm~\cite{Cacciari:2008gp} with a distance parameter of $0.4$.
%
The measurements are reported in terms of kinematic observables defined before the decay of the Higgs boson, i.e. at the generator level.


% ____________________________________________________________________________
\subsection{\texorpdfstring{$\hgg$}{H to gamma gamma}}

\emph{%
Unless otherwise indicated, all information in this section originates from Ref.~\cite{Sirunyan:2018kta}. This section concerns merely a summary of the full analysis; for more details, see the full documentation.
}

Because of the excellent photon energy resolution of the CMS detector, the $\hgg$ decay channel is a clean decay channel with an excellent mass resolution.
% 
With its good signal-to-background ratio and good energy resolution, it was the driving input for the discovery of the Higgs boson in 2012~\cite{Aad:2012tfa,Chatrchyan:2012xdj,Chatrchyan:2013lba}, despite its relatively small branching fraction.


The analysis strategy of the $\hgg$ analysis revolves around the measurement and prediction of the diphoton mass $\mgg$ spectrum---the summed invariant mass of two reconstructed photons.
% 
As discussed in Section~\ref{sec:ecal}, the CMS detector has an excellent photon energy resolution, which translates to a good mass resolution.
% 
The shape of the $\mgg$ spectrum is characterized by a peak around 125\GeV, on top of a smooth falling background consisting of an irreducible QCD contribution, and a reducible background of hadrons misidentified as photons (typically $\pi_0$s).
% 
The irreducible QCD background, which at leading order concerns two gluons decaying to two photons via a quark loop, is exponentially falling for higher values of $\mgg$.
% 
As this process concerns real photons, it cannot be suppressed.
% 
The second background, concerning hadrons misidentified as photons, can be effectively reduced by discriminating on information about the candidate from ECAL, beyond simply its energy measurement.
% 
Distinguishing real photons from misidentified ones is done via a boosted decision tree (BDT), a machine-learning-based approach to fit a complicated multivariate function.
% 
The most important variables that are used as input to the photon identification BDT are those pertaining to the \textit{shower shape} of the candidate, i.e. the profile of the energy deposit in ECAL.
% 
Other important variables include the degree of isolation (i.e. lack of other candidates around the candidate) and the location in the detector.


Globally, a number of loose event selections are applied in order to avoid events in poorly understood regions of the detector, and to avoid regions that are clearly of no interest for the intended signal.
% 
Firstly, events are selected with deposits in ECAL at $\eta<2.5$, up to which the detector inefficiency is still negligible, and events with deposits in the region $1.4442 < \eta < 1.566$, which is the difficult-to-model transition region from the barrel to the end cap.
% 
Secondly, only events with a leading (subleading) photon candidate with $\pt > 30$ (20)\GeV are considered.
% 
Lastly, a number of loose selection criteria on the shower profiles of the photon candidates are required: These criteria make sure the photon candidates are contained in a sufficiently small cone and are sufficiently isolated.


In order to reduce the model dependence of the measurement, the $\hgg$ analysis is performed in a \textit{fiducial phase space}.
% 
The fiducial phase space is typically a part of the full phase space in which, from the experimental side, the detector is well understood, and, from a theoretical side, the theoretical uncertainties are small.
% 
As it is unlikely that our understanding of the detector or the theoretical calculations in this phase space will change radically, these requirements ensure the validity of the final results on the long term.
% 
The fiducial phase space for the $\hgg$ analysis is defined by first requiring that ratio of the leading (subleading) photon $\pt$ over $\mgg$ is greater than $1/3$ ($1/4$).
% 
Secondly, a degree of purity of the photon is required, which is implemented by requiring that the scalar sum of the generator-level $\pt$ of stable particles contained in a cone of radius $\Delta R=0.3$ around the candidate is less than 10\GeV, where $\Delta R = \sqrt{\smash[b]{(\Delta\eta)^2+(\Delta\phi)^2}}$ is the angular separation between particles and $\Delta\phi$ is the azimuthal angle between two particles in radians.
% 
These requirements are imposed at the generator level, and as the detector resolution is finite, small contributions from outside the fiducial phase space, termed the \textit{outside-of-acceptance} region, may contribute to the event counts in the fiducial phase space.
% 
Likewise, events inside the fiducial phase space may be reconstructed outside of it due to finite detector resolution.
% 
A similar effect occurs among bins of the differential variable; events originating from one bin may be reconstructed in another.
% 
This effect is referred to as \textit{bin-to-bin migrations}.
% 
Both effects are modeled and included the differential measurement.


Rather than a single $\mgg$ spectrum for all selected events, the events are divided into mutually exclusive \textit{categories} with varying signal-to-background ratios, in order to enhance the sensitivity of the analysis.
% 
The categorization applied for the input used in this thesis concerns first a split between the $\ggh$ production mode and all other modes; secondly, the events are split into three categories of varying diphoton mass resolution, $\sigma_{\mgg}$.
% 
Events are then further split up into bins of the differential variable of interest.
% 
This division of events yields one $\mgg$ spectrum to be fitted per production mode, category, and bin of the differential variable.
% 
Examples of the (unfitted) signal and background distributions in the $\mgg$ spectrum, including the actual event counts in the 2016 dataset, are shown in Fig.~\ref{fig:example_mgg}.
% 
Note that the signal includes contributions from the outside-of-acceptance region and from inefficiencies of other $\pt$ bins.

\begin{figure}[hbtp]
  \begin{center}
    \includegraphics[width=0.7\linewidth]{img/inputs/hgg/example_mgg.png}
    \caption{
        Example of the $\mgg$ spectrum.
        % 
        The blue line shows an example of a background distribution, the red lines indicate the signal distributions in addition to the background distribution for 0.5, 1.0, and 2.0 times the SM distribution, and the black dots indicate the actual event counts in the 2016 dataset.
        }
    \label{fig:example_mgg}
  \end{center}
\end{figure}


Because of its comparatively high event count at higher transverse momentum, the $\hgg$ decay channel is the most finely binned one in terms of $\pth$, and is the leading contribution to the overall precision of the combination.
% 
Some of the results obtained in Ref.~\cite{Sirunyan:2018kta} are shown in Fig.~\ref{fig:hgg-results}.


\begin{figure}[hbtp]
  \begin{center}
    \includegraphics[width=0.49\linewidth]{img/inputs/hgg/pth.pdf}
    \includegraphics[width=0.49\linewidth]{img/inputs/hgg/absy.pdf}
    \\
    \includegraphics[width=0.49\linewidth]{img/inputs/hgg/njets.pdf}
    \includegraphics[width=0.49\linewidth]{img/inputs/hgg/ptjet.pdf}
    \caption{
        Differential measurements of the $\pth$ (top left), $\absy$ (top right), $\njets$ (bottom left), and $\ptjet$ (bottom right) spectra from the analysis of the $\hgg$ decay channel.
        % 
        Taken from Ref.~\cite{Sirunyan:2018kta}.
        }
    \label{fig:hgg-results}
  \end{center}
\end{figure}



\tk{Maybe move to a global ``Combination'' section, including the binning tables.} With respect to the analysis performed in Ref.~\cite{Sirunyan:2018kta}, a few adaptations were made in order to facilitate the combination of differential cross sections performed in Section~\ref{sec:diffxs}.
% 
Firstly, the signal contributions were split into contributions from only $\ggh$ and other production modes.
% 
Secondly, the binning of the differential variables was modified slightly: The $\pth$ spectrum contains an additional bin, $\pth>600$\GeV, and an additional bin boundary at $\absy = 1.2$ is included in the $\absy$ spectrum.



% ____________________________________________________________________________
\subsection{\texorpdfstring{$\hzztofourl$}{H to ZZ to four leptons}}

\emph{%
Unless otherwise indicated, all information in this section originates from Ref.~\cite{Sirunyan:2017exp}. This section concerns merely a summary of the full analysis; for more details, see the full documentation.
}

Due to the good energy resolution of leptons in the CMS detector, the $\hzztofourl$ decay channel has an exceptionally clean final state, combined with a good signal-to-background ratio.
% 
Its only crutch is the low branching fraction; the first stage of the decay, $\hzz$, has a branching fraction of only $0.26\%$ at $\mh = 125.09\GeV$~\cite{deFlorian:2016spz}; combined with the second stage, requiring the decays of the $\zboson$ bosons to either electrons or muons, reduces the branching fraction further to $1.251 \times 10^{-2}$ \%~\cite{deFlorian:2016spz}.
% 
Despite its low branching fraction, the differential measurements performed in the $\hzz$ decay channel have a good precision at low $\pt$; at higher $\pt$ the event count becomes very small and the precision deteriorates.




% ____________________________________________________________________________
\subsection{\texorpdfstring{$\hbb$}{H to bb}}

\tk{TODO}


% \subsection{Paper text}

% For all the analyses used as input to the combination ($\hgg$~\cite{Sirunyan:2018kta}, $\hzztofourl$~\cite{CMS_AN_2016-442}, and $\hbb$~\cite{CMS_AN_2016-366}), the data set corresponds to an integrated luminosity of $35.9\fbinv$ recorded by the CMS experiment in 2016.
% The $\hbb$ decay channel is only included in the combination of the $\pth$ spectra, improving the measurements at the higher end of the distribution ($\pth > 350$\GeV) where the data from the $\hgg$ and $\hzz$ decay channels are limited.
% All analyses provide the parametrization of the folding matrix $M_{ji}^{k}$ (which is the probability for an event in generator-level bin $i$ to be reconstructed in bin $j$ and category $k$) in terms of a common generator-level binning, that is used for the combined spectra.
% Given the limited statistical precision in the individual channels, the results of the $\hzz$ and $\hbb$ channels individually are reported for a coarser binning, which is provided in Tables~\ref{tab:binningpth}--\ref{tab:binningptjet} for each of the observables.
% This binning coincides with the binning at the reconstruction level.


% The SM prediction for the differential cross sections is simulated with $\MGvATNLO$ v2.2.2~\cite{Alwall:2014hca} for each of the four dominant Higgs boson production modes: gluon-gluon fusion (\ggh), vector boson fusion, associated production with a $\wboson$/$\zboson$ boson, and associated production with a top quark-antiquark pair.
% A contribution from Higgs boson production in association with bottom quarks is not simulated, but included assuming its acceptance is equal to that from Higgs boson production via gluon fusion.
% The matrix element calculation includes the emission of up to two additional partons and is performed at NLO accuracy in perturbative quantum chromodynamics (QCD).
% Events are interfaced to \PYTHIA8.205~\cite{Sjostrand:2014zea} for parton showering and hadronization with the CUETP8M1~\cite{Skands:1695787} underlying event tune.
% The matrix element calculation is matched to the parton shower following the prescription in Ref.~\cite{Frederix:2012ps}.
% A weight depending on $\pth$ and $\njets$ is applied to simulated $\ggh$ events to match the predictions from the {\textsc{nnlops}} program~\cite{Hamilton:2012np, Kardos:2014dua}, as discussed in Ref.~\cite{Sirunyan:2018koj}.
% The set of parton distribution functions used in all simulations is NNPDF3.0~\cite{Ball:2014uwa}.
% The hadronic jets are clustered from the particle-flow candidates~\cite{Sirunyan:2017ulk} in the case of data and simulation, and from stable particles excluding neutrinos in the case of generated events, using the anti-$\kt$ clustering algorithm~\cite{Cacciari:2008gp} with a distance parameter of $0.4$.
% The measurements are reported in terms of kinematic observables defined before the decay of the Higgs boson, i.e. at the generator level.


% Each of the analyses used as input to the combination corresponds to a different fiducial phase space definition and applies a different event categorization.
% In the case of the $\hgg$ analysis, the fiducial phase space is defined by requiring the ratio of the leading (subleading) photon $\pt$ to the diphoton mass to be greater than $1/3$ ($1/4$).
% In addition, for each photon candidate the scalar sum of the generator-level $\pt$ of stable particles contained in a cone of radius $\Delta R=0.3$ around the candidate is required to be less than 10\GeV, where $\Delta R = \sqrt{\smash[b]{(\Delta\eta)^2+(\Delta\phi)^2}}$ is the angular separation between particles and $\Delta\phi$ is the azimuthal angle between two particles in radians.
% The selected photon pairs are categorized according to their estimated relative invariant mass resolution~\cite{Sirunyan:2018kta}.
% In the case of the $\hzz$ analysis, the 4-lepton mass is required to be greater than 70\GeV, the leading $\zboson$ boson candidate invariant mass must be greater than 40\GeV, and leptons must be separated in angular space by at least $\Delta R > 0.02$.
% Furthermore, at least two leptons must each have a $\pt>10$\GeV and at least one a $\pt > 20$\GeV.
% The selected events are categorized according to their lepton configuration in the final state (4 electrons, 4 muons, or 2 electrons and 2 muons).
% In the case of the $\hbb$ analysis, the analysis strategy requires the presence of a single anti-$\kt$ jet with a distance parameter of $0.8$, $\pt>450\,$GeV, and $\abs{\eta}<2.5$.
% For this analysis, the data is not unfolded to a fiducial phase space.
% Soft and wide-angle radiation is removed using the soft-drop grooming algorithm~\cite{Dasgupta:2013ihk,Larkoski:2014wba}.
% The jet mass after application of the soft-drop algorithm, $\msd$, peaks close to the Higgs boson mass in the case of signal events.
% To avoid finite-cone effects and the nonperturbative regime of the $\msd$ calculation, events are selected based on the dimensionless mass scale variable for QCD jets defined as $\rho=\log\left(\msd^2/\pt^2\right)$~\cite{Dasgupta:2013ihk}, which relates the jet $\pt$ to the jet mass.
% Events with isolated electrons, muons, or \taulepton leptons with $\pt>10$\GeV and $\abs{\eta}<2.5$ are vetoed in order to reduce the background from SM electroweak processes, and events with a missing transverse momentum greater than $140$\GeV are vetoed in order to reduce the background from top quark-antiquark pair production.
% Additionally, a selection criterion is applied based on the compatibility of the single anti-$\kt$ jet with having a two-prong substructure~\cite{Dolen:2016kst,Moult:2016cvt,Larkoski:2013eya,Thaler:2010tr}.
% Events are categorized according to their likelihood of consisting of two $\bquark$ quarks, which is computed using the double-$\bquark$ tagger algorithm~\cite{Sirunyan:2017ezt}.


% Minor modifications are applied to the individual analyses in Refs.~\cite{Sirunyan:2018kta,CMS_AN_2016-442,CMS_AN_2016-366} to provide the inputs used for the combination of differential observables.
% For $\hgg$, an additional bin, $\pth>600$\GeV, is included in the $\pth$ spectrum.
% For $\hzz$, the binning is modified for multiple kinematic observables to align with the binning of the $\hgg$ analysis.
% Furthermore, the branching fractions of the two $\zboson$ bosons to the various lepton configurations are fixed to their SM values, whereas in Ref.~\cite{CMS_AN_2016-442} these are allowed to float.
% For $\hbb$ the signal is split into two $\pt$ bins at the generator level:
% the first with $350\le\pt<600$\GeV and the second an overflow bin with $\pt\ge600$\GeV, which aligns with the binning of the other channels.
% At the reconstruction level two bins are employed, with $400\le\pt<600$ and $\pt\ge600$\GeV, which is a slight modification with respect to the binning used in Ref.~\cite{CMS_AN_2016-366}.
% The redefinition of the reconstructed $\pt$ categories necessitates a reevaluation of the background model, which is performed using the same procedure as in the original analysis.
% For the purpose of the combination in this analysis, the fiducial measurements from the $\hgg$ and $\hzz$ channels are extrapolated to the inclusive phase space~\cite{Alwall:2014hca,Hamilton:2012np,Kardos:2014dua}.


% \begin{table*}[htb]
%     \centering
%     \topcaption{
%         The reconstruction-level binning for $\pth$ for the $\hgg$, $\hzz$, and $\hbb$ channels.
%         This binning coincides with the binning of the unfolded cross sections in which the individual results are reported.
%         }
%     \label{tab:binningpth}
%     \tabletextwidth{
%     \setlength{\tabcolsep}{5pt}
%     \begin{tabular}{lccccccccc}
%     Channel & \multicolumn{9}{l}{$\pth$ binning (GeV)} \\[\tablelineskip]
%     \hline
%     $\hgg$
%         & [0, 15)    & [15, 30)   & [30, 45)   & [45, 80)        & [80, 120)
%         & [120, 200) & [200, 350) & [350, 600) & [600, $\infty$)
%         \\
%     $\hzz$
%         & [0, 15) & [15, 30)
%         & \multicolumn{2}{l}{[30,  80)}
%         & \multicolumn{2}{l}{[80,  200)}
%         & \multicolumn{3}{l}{[200, $\infty$)}
%         \\
%     $\hbb$
%         & \multicolumn{7}{@{{}}c@{{}}}{None} & [350, 600) & [600, $\infty$)
%         \\
%     \end{tabular}
%     }
%     \end{table*}

% \begin{table}[htb]
%     \centering
%     \topcaption{
%         The binning for $\njets$ for the $\hgg$ and the $\hzz$ channels.
%         This binning coincides with the binning of the unfolded cross sections in which the individual results are reported.
%         }
%     \label{tab:binningnjets}
%     \begin{tabular}{lccccc}
%     Channel & \multicolumn{5}{l}{$\njets$ binning} \\[\tablelineskip]
%     \hline
%     $\hgg$ & 0 & 1 & 2 & 3 & $\ge$4 \\
%     $\hzz$ & 0 & 1 & 2 & \multicolumn{2}{l}{$\ge$3} \\
%     \end{tabular}
%     \end{table}

% \begin{table*}[htb]
%     \centering
%     \topcaption{
%         The binning for $\absy$ for the $\hgg$ and the $\hzz$ channels.
%         This binning coincides with the binning of the unfolded cross sections in which the individual results are reported.
%         }
%     \label{tab:binningabsy}
%     \begin{tabular}{lcccccc}
%     Channel & \multicolumn{6}{l}{$\absy$ binning} \\[\tablelineskip]
%     \hline
%     $\hgg$ & [0.0, 0.15) & [0.15, 0.30) & [0.30, 0.60) & [0.60, 0.90) & [0.90, 1.20) & [1.20, 2.50] \\
%     $\hzz$ & [0.0, 0.15) & [0.15, 0.30) & [0.30, 0.60) & [0.60, 0.90) & [0.90, 1.20) & [1.20, 2.50] \\
%     \end{tabular}
%     \end{table*}

% \begin{table*}[h!]
%     \centering
%     \topcaption{
%         The binning for $\ptjet$ for the $\hgg$ and the $\hzz$ channels.
%         This binning coincides with the binning of the unfolded cross sections in which the individual results are reported.
%         }
%     \label{tab:binningptjet}
%     \begin{tabular}{lcccccc}
%     Channel & \multicolumn{6}{l}{$\ptjet$ binning (GeV)} \\[\tablelineskip]
%     \hline
%     $\hgg$ & [0, 30) & [30, 55) & [55, 95) & [95, 120) & [120, 200) & [200, $\infty$) \\
%     $\hzz$ & [0, 30) & [30, 55) & [55, 95) & \multicolumn{3}{l}{ [95, $\infty$) } \\
%     \end{tabular}
%     \end{table*}

