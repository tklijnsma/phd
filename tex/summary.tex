\section{Summary}

While the standard model (SM) has been remarkably successful in describing and predicting contemporary high energy physics observations, it cannot provide the full picture of particle physics, and new physics will start playing a role at some currently unknown energy scale.
% 
A major part of the current and future LHC physics analysis work will concern the search for deviations from the SM predictions, in terms of newly discovered particles or in terms of deviations in precision measurements of physics model parameters.
% 
Regarding the former, the SM has proven to be `disappointingly' resilient for anyone who hoped to explore a whole new world of particles after the discovery of the Higgs boson.
% 
Supersymmetry, one of the neatest theoretical extensions of the SM, has so far been undiscovered, and with every new search result published, the excluded phase space grows ever larger.
% 
Regarding the latter, the Higgs boson couplings may be one of the key sets of parameters to find deviations, as the properties of the Higgs boson are subject to alterations in many types of beyond-the-SM physics models.
% 
The latest measurements of the Higgs boson couplings, conducted by the CMS~\cite{Sirunyan:2018koj} and ATLAS~\cite{ATLAS-CONF-2019-005} Collaborations using inclusive cross sections, are in striking agreement with the SM predictions---there are almost no deviations larger than 1 standard deviation.
% 
With the advent of much more data, but (at least on the short term) a not much higher center-of-mass energy $\sqrt{s}$, the field of Higgs physics enters a precision era, in which more complex techniques yielding good constraints on physics model parameters may provide the key to finding deviations from the SM.


The potential of differential cross sections as a probe of Higgs boson properties, and in particular its couplings to other particles, is not a new idea, but only recently enough data has been collected in order to measure differential cross sections with enough precision for this purpose.
% 
One of the first studies on constraining Higgs boson couplings using measured differential Higgs boson production cross sections was performed in Ref.~\cite{Bishara:2016jga}.
% 
Here, constraints were set for the Higgs boson coupling modifier to charm quarks $\kappac$ and to bottom quarks $\kappab$, based on a measurement of the Higgs boson transverse momentum $\pth$ spectrum by the ATLAS Collaboration~\cite{Aad:2015lha} using data collected at $\sqrt{s}=8$\TeV corresponding to an integrated luminosity of $20.3$\fbinv.
% 
A second model~\cite{Grazzini:2017szg,Grazzini:2016paz} involving simultaneous variations of the Higgs boson coupling modifier to top quarks $\kappat$, an anomalous direct coupling of the Higgs boson to the gluon field $\cg$, and $\kappab$ is promising, as it is able to resolve finite quark mass effects in the differential gluon fusion production mode ($\ggh$), which is nearly impossible to do in an inclusive measurement~\cite{Azatov:2013xha}.


This thesis treated the measurement and interpretation of the differential Higgs boson production cross sections.
% 
The first part of this thesis concerned the combination of differential cross section measurements from several decay channels within the CMS Collaboration: the Higgs boson decaying to two photons ($\hgg$), four leptons via two $\zboson$ bosons ($\hzztofourl$), and a bottom-antibottom quark pair in a boosted topology ($\hbb$).
% 
Particular challenges to overcome were aligning the generator-level binning between decay channels for all the observables, taking into account the correlation structure of systematic uncertainties between decay channels, treating the acceptance uncertainties, and many more.
% 
The results of the combination shown in Section~\ref{sec:diffxs-results} have a markedly improved precision with respect to results from any individual decay channel.
% 
The precision on the inclusive production cross section measurement was improved by nearly 25\% with respect to the $\hgg$ result individually.
% 
Regarding differential cross sections, the combination was performed for four differential observables: $\pth$, the jet multiplicity $\njets$, the $\pt$ of the leading jet $\ptjet$, and the rapidity of the Higgs boson $\absy$.
% 
No significant deviations from the SM prediction are observed in any of the combined spectra.
% 
The precision on the $\pth$ spectrum, the most important one for interpretation purposes, was improved by 15\% with respect to the $\hgg$ analysis alone; the improvement is particularly noticeable in the low-$\pth$ regions where there is enough data in the $\hzz$ channel, and in the very tail of the distribution, where the $\hbb$ channel is sensitive.
% 
In Section~\ref{sec:projections}, the $\pth$ spectrum was projected to an integrated luminosity of $3000\fbinv$, the total expected integrated luminosity of the High-Luminosity LHC.
% 
The $\pth$ spectrum is expected to have an uncertainty per bin of about $5\%$ (assuming the binning does not change), increasing up to an uncertainty of about $25\%$ in the very last bin, $\pth > 600\GeV$.
% 
The projected uncertainty on this last bin is of relatively high importance for the theory community, as many theoretical models expect deviations from the SM at large $\pth$.


The second part of this thesis treated the interpretation of the $\pth$ spectrum in terms of the Higgs boson couplings, using the theoretical models built in Refs.~\cite{Grazzini:2017szg,Grazzini:2016paz,Bishara:2016jga}.
% 
Both models concern an effective field theory, in which the SM Lagrangian is extended with higher dimensional terms.
% 
The implementation of the theoretical $\pth$ spectrum in the extended likelihood proved to be a significant technical challenge.
% 
It required first an accurate parametrization of the $\pth$ spectrum under various Higgs boson coupling variations, a scheme for dealing with bin-to-bin correlations of the theoretical uncertainties, and a careful consideration of the constraints that stem from the inclusive cross section rather than from the shape of the $\pth$ distribution.
% 
The results for simultaneous variations of $\kappat$, $\cg$, and $\kappab$, shown in Section~\ref{sec:interpretation-results-ktcgkb}, show that the measurements indicate the presence of finite quark mass effects, rather than an infinite top mass in the loop structure of $\ggh$; while not unexpected, this is the first pieces of experimental evidence in this regard.
% 
The results for simultaneous variations of $\kappab$ and $\kappac$, shown in Section~\ref{sec:interpretation-results-kbkc}, provide constraints on $\kappac$ that are largely independent from other $\kappac$ determinations.
% 
Using only the shape of the $\pth$ distribution (i.e.\ profiling the overall normalization of the spectrum), one finds $\kappacLeftObservedFLOATINGBRS < \kappac < \kappacRightObservedFLOATINGBRS$ at 95\% \CL; while not the most competitive constraints, this fit on differential cross sections is worth including in a potential future $\kappac$ combination.
% 
At a projected integrated luminosity of $3000\fbinv$, these limits improve to roughly $\abs{\kappac}\lesssim9$.


In the third and final part of this thesis, the strong coupling constant $\alphas$ was determined using top-antitop quark pair production cross section ($\stt$) measurements from the CMS and ATLAS Collaborations, and from the combination of measurements at the Tevatron.
% 
This determination is one of the few available at next-to-next-to-leading (NNLO) order in perturbative quantum chromodynamics at hadron colliders, and is therefore mostly uncorrelated from other $\alphas$ determinations that drive the world average computed by the Particle Data Group~\cite{pdg}.
% 
The determination procedure is largely based on the one applied in Ref.~\cite{Chatrchyan:2013haa}, which was based on a $\stt$ measurement performed by the CMS Collaboration at $\sqrt{s}=7\TeV$.
% 
The second stage of this work, after applying the determination procedure on the individual cross section measurements, concerned the combination of these individual $\alphas$ determinations.
% 
The combination was carried out using an extended likelihood, based on the same principles as the one used for the combination of differential cross sections.
% 
Throughout this work, several technical complications (e.g. implementing a partial correlation structure in the likelihood) were overcome and difficult decisions (e.g. which PDF sets to use, and whether to use the resummation scheme or not) were made.
% 
The final results is $\asmz = \alphasResultCenter^{+\alphasResultRightError}_{-\alphasResultLeftError}$.

