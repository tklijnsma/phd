\clearpage
\section{Energy regressions for photons and electrons in the 2016 dataset}
\label{app:regression}

\newcommand{\Eraw}{\ensuremath{E_\text{raw}}\xspace}
\newcommand{\Etrue}{\ensuremath{E_\text{true}}\xspace}
\newcommand{\Ecorr}{\ensuremath{E_\text{corr}}\xspace}

The energy of reconstructed photons is measured by translating the detected amount of light in the scintillating lead-tungstate crystals into energy measurements, and summing the energy of crystals within a specified region called the \textit{supercluster}.
% 
This energy measurement yields what is called the \textit{raw energy} $\Eraw$; it is the baseline for further corrections, which are designed to approximate the \textit{true energy} $\Etrue$ as closely as possible.
% 
These corrections are required mostly to solve the lack of containment of the particle's energy deposit within CMS ECAL.
% 
For example, gaps in between the crystals and modules may cause some deposited energy not to be reconstructed, a problem referred to as limited \textit{local containment}.
% 
In addition, there is a non-negligible material budget in between the interaction point and ECAL, leading to photon conversions and brehmstrahlung for electrons.
% 
The radiated particles from these effects are typically bent along the $\phi$ direction due to the strong magnetic field, and may on occasion fail to be included in the determination of the supercluster~\cite{Chatrchyan:2013dga}.
% 
This problem is referred to as limited \textit{global containment}.
% 
The procedure of obtaining an estimate of $\Etrue$ from $\Eraw$ is called the \textit{energy regression}.


The regression is carried via a \textit{semiparametric boosted decision tree (BDT)}, which is a boosted decision tree that in its leafs carries a parametrization of the target.
% 
This type of multivariate analysis is called semiparametric as the product is parametric, but the decision tree itself is not.
% 
The parametric form chosen in the leafs is a so-called \textit{double-sided crystal ball function} (DCSB) function, with the following formulaic form~\cite{Aad:2015oqa}:
% 
\begin{linenomath*}
\begin{equation}
\text{DCSB}(x) = \left\{
    \begin{array}{ll}
    \exp{-\alpha_L^2/2}
        \left( 1-\frac{\abs{\alpha_L}}{n_L} \left(\abs{\alpha_L}+t\right) \right)^{-n_L}
        & \text{if} \quad t \leq -\alpha_L, \\
    % 
    \exp{-t^2/2} & \text{if} \quad -\alpha_L < t < \alpha_R, \\
    % 
    \exp{-\alpha_R^2/2}
        \left( 1-\frac{\abs{\alpha_R}}{n_R} \left(\abs{\alpha_R}+t\right) \right)^{-n_R}
        & \text{if} \quad t \geq \alpha_R, \\
    \end{array}
    \right.
\end{equation}
\end{linenomath*}
% 
where $t = \frac{x-\mu}{\sigma}$, and $\mu$, $\sigma$, $\alpha_{L/R}$ and $n_{L/R}$ are parameters of the DSCB; importantly, $\mu$ and $\sigma$ are the mean and standard deviation of the Gaussian core of the function.
% 
For the regression performed here, $\alpha_L$ has been fixed to 2, and $\alpha_R$ to 1; these constant values seem to work well throughout the energy regime, and fixing them allows the BDT to learn the more important parameters of the Gaussian core.
% 
The regression takes as input a number of variables from ECAL, such as the raw energy, the location of the supercluster in the detector, the width and height of the supercluster in terms of $\eta$ and $\phi$, and the ratio of the total energy measured in HCAL over the total energy measured in ECAL.
% 
The most important variables, however, pertain to the \textit{shower shape} of the electron or photon; notable examples include \textit{R9}, defined as the ratio of the summed energy of the $3\times3$-cluster around the electron/photon over the energy of the supercluster, and ratios of various crystal arrays over the summed energy of the $5\times5$-cluster around the electron/photon.
% 
In the case of electrons, a second set of variables is available from the tracker; these are mostly important for low-energy electrons.
% 
The regression is performed separately for the barrel and endcap regions, and as such the input variables also differ slightly; for example, in the endcap regions the preshower energy measurement is included as an input, and rather than $\eta$ and $\phi$, the location of the superclusters are given in planar coordinates.


For low-energy electrons, the most prominent variables for the energy correction are those from the tracker.
% 
While an obvious fact from physics concerns, in a BDT these type of trivialities must be learned through the analysis of large amounts of data.
% 
In order to speed up the training, increase chances of convergence and to improve the final result, it is typically preferable to design the training with these given facts in mind.
% 
The energy regression for electrons is determined by first training a BDT using only ECAL variables.
%
As a target function, which is to be brought as closely to 1 as possible with minimal standard deviation per leaf, it uses
% 
\begin{linenomath*}
\begin{equation}
\text{target}_1 = \frac{\Etrue}{\Eraw}
\,.
\end{equation}
\end{linenomath*}
% 
The corrected energy $\Ecorr$ that is yielded by this BDT is already a good estimate of $\Etrue$, and for energies $>100\GeV$ it is basically the final estimate.
% 
Subsequently, a second BDT is trained, this time including the tracker variables, with the target defined as
% 
\begin{linenomath*}
\begin{equation}
\text{target}_2 = 
% 
\frac{\Etrue}{E_\text{estimate}} =
% 
\frac{\Etrue}
{
    \left( \frac{\sigma^2_{\abs{p}}}{\sigma^2_{\abs{p}} + \sigma^2_{\Ecorr}} \right) \Ecorr
    +
    \left( \frac{\sigma^2_{\Ecorr}}{\sigma^2_{\abs{p}} + \sigma^2_{\Ecorr}} \right) \abs{p}
}
\,,
\end{equation}
\end{linenomath*}
% 
where $\abs{p}$ is the absolute momentum determined from the tracker and $\sigma_{\abs{p}}$ its uncertainty.
% 
The denominator is essentially a weighted sum of the energy estimations from the tracker and from ECAL, where the weights are determined by the respective uncertainties.
% 
With this two-tiered approach, the regression has only to learn in the second stage to trust the tracker energy estimate more for low-energy electrons, and the ECAL estimate for high-energy electrons, greatly reducing the required training time and improving the chances for convergence.


\tk{Todo: Saturation and Results.}








