\subsection{Introduction}

The strong coupling constant of Quantum Chromodynamics (QCD), $\as$,
is, together with the quark masses, the main free parameter of the QCD
Lagrangian. 
%
It enters into every process that involves the strong interaction and
is the fundamental parameter of the perturbative expansion used in
calculating cross sections for processes with large momentum
transfers.

The strong coupling is a function of a renormalisation scale $\mu$.
%
Its dependence on $\mu$ is governed by renormalisation group
equations~\cite{Baikov:2016tgj,Herzog:2017ohr}, however its value at a given reference
scale must be determined from experimental data.
%
The current world average value for the coupling evaluated at the
$Z$-boson mass scale, $\asmz$, as determined by the Particle Data
Group (PDG), is $0.1181 \pm 0.0011$~\cite{pdg}.
%
The world average incorporates information from a wide variety of
experimental data and of methods to deduce $\as$ from that data.
%
It requires at least next-to-next-to-leading order (NNLO) accuracy in
the perturbative expansions that are used.

Even with the $1\%$ precision that is quoted by the PDG, the
uncertainty on $\as$ contributes significantly to uncertainties on
physical predictions for colliders.
%
For example, it leads to about $2\%$ uncertainty on the gluon-fusion
Higgs cross section, comparable with the largest of any of the other
individual uncertainties~\cite{Anastasiou:2016cez}.
% 
Furthermore, while the bulk of the evidence points to values of the
strong coupling that are compatible with $\asmz \simeq 0.118$,
including precise lattice-QCD based
determinations, e.g.~\cite{pdg,Aoki:2016frl,McNeile:2010ji,Bruno:2017gxd},
%
there are a handful determinations with small quoted uncertainties
that suggest $\asmz$ values that are several standard deviations below
the world average. 
%
Notable cases are those from the
Thrust and C-parameter distributions in $e^+e^-$ collisions, which
yield $0.1135 \pm 0.0011$ and $0.1123 \pm 0.0015$
respectively~\cite{Abbate:2010xh,Hoang:2015hka},\footnote{ An
  alternative analysis of the Thrust quotes a significantly larger
  uncertainty, $0.1137^{+0.0034}_{-0.0027}$~\cite{Gehrmann:2012sc}.}
%
or the ABMP PDF fit~\cite{Alekhin:2017kpj}, $0.1147\pm0.0008$.

Of the various NNLO determinations of the strong coupling, so far only one is
based on hadron collider data, using a measurement of the top-quark
pair production cross section ($\stt$) performed by the CMS
Collaboration at a centre-of-mass energy
$\sqrt{s}=7\,$TeV~\cite{Chatrchyan:2013haa}.
%
It yields $\asmz = 0.1151^{+0.0028}_{-0.0027}$.
%
This extraction is intriguingly placed between the world average and
the outlying low $\as$ extractions, albeit compatible with both.
%
However, it is based on a single, early and now outdated measurement of $\stt$.
%
It is of interest, therefore, to examine how it is affected by more
recent precise measurements by the ATLAS and CMS Collaborations at
CERN's Large Hadron Collider (LHC)
\cite{Khachatryan:2016mqs,Khachatryan:2015uqb,Sirunyan:2017uhy,Aad:2014kva,Aaboud:2016pbd}
as well as by a combination of measurements from the D0 and CDF
collaborations at the Tevatron~\cite{Aaltonen:2013wca}.

In the course of our discussion, we will encounter issues related to
the treatment of theoretical uncertainties and the choice of the
parton distribution function (PDF) set
that are of relevance more generally in the determination of the
strong coupling and other fundamental constants (e.g.\ the top-quark
mass) from collider data.
%
Such studies may become increasingly widespread in the coming
years, given the recent rapid progress in NNLO calculations, e.g.\ for
vector-boson (e.g.\ Refs.~\cite{Boughezal:2015ded,Ridder:2016nkl}) and
inclusive jet $p_t$ 
distributions~\cite{Currie:2017tfd} at hadron colliders and jet $p_t$
distributions in Deep Inelastic Scattering (DIS)~\cite{Currie:2017tpe}.
