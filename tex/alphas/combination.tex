\subsection{Combination of $\as$ determinations}
\label{sec:combination}

\subsubsection{Correlation coefficients}
\label{sec:correlationcoefficients}

A combination of measurements can strongly depend on the assumed or calculated correlations~\cite{BLUE1}.
%
It is therefore necessary to carefully evaluate the correlation coefficients used for the combination.
In the case of $\as$ determinations many correlations can be reasonably motivated or computed.
The correlation coefficients between individual measurements are motivated per uncertainty source.

\newcommand{\lowersub}[1]{\raisebox{-3pt}{\scriptsize #1}}
\newcommand{\highersub}[1]{\raisebox{2pt}{\scriptsize #1}}

\begin{enumerate}
\item \textit{Statistical uncertainties} are considered uncorrelated
  for all experimental inputs.

\item \textit{Systematic uncertainties} are considered fully
  correlated only for measurements obtained with the same
  detector. This concerns the measurements performed by CMS and ATLAS
  at different centre-of-mass energies.

\item \textit{Uncertainties due to beam energy} are fully correlated
  between ATLAS and CMS and are taken to be correlated across
  energies.
  %
  The beam-energy uncertainty at the Tevatron was tiny and is
  neglected, as outlined in the caption of
  Table~\ref{tab:includedmeasurements}. 
  %

\item
    \textit{Uncertainties due to luminosity} are partially correlated between
    ATLAS and CMS. The correlated component of the luminosity uncertainty
    stems from the uncertainty on the bunch current density and
    similarities in the Van der Meer scan fit model.

    The correlated and uncorrelated uncertainties are estimated using the same
    principles as used for the top-quark pair production cross section combinations
    between ATLAS and CMS at 7 and
    8\,TeV~\cite{topcombination_7TeV,topcombination_8TeV}, updated
    with the latest luminosity
    determinations~\cite{Aad:2013ucp,Aaboud:2016hhf,CMS:2012rua,CMS:2013gfa,CMS:2016eto}.
    %
    The luminosity uncertainty (as a percentage of the top-quark pair
    production cross section) is displayed in Table~\ref{tab:lumitable}.
    The luminosity uncertainties on $\as$ are taken to have the same correlation
    coefficient.

    \newcommand{\tableWidthLumi}{2.5cm}
    \begin{table*}[ht]
    \centering
    \topcaption{%
        Correlated, uncorrelated and total luminosity uncertainties with
        respect to the top-quark pair production cross section (in
        percentages)~\cite{topcombination_7TeV,topcombination_8TeV,Aad:2013ucp,Aaboud:2016hhf,CMS:2012rua,CMS:2013gfa,CMS:2016eto}.
        }
    \label{tab:lumitable}
    % 
    \begin{tabular}{l c c c c}
    $\sqrt{s}$                & Experiment & Corr.  & Uncorr. & Total \\
    \midrule
    \multirow{2}{*}{7\,TeV}   &  ATLAS     &  0.46\%  &  1.72\%  &  1.78\% \\
                              &  CMS       &  0.46\%  &  2.13\%  &  2.17\% \\[7pt]
    \multirow{2}{*}{8\,TeV}   &  ATLAS     &  0.60\%  &  1.84\%  &  1.94\% \\
                              &  CMS       &  0.68\%  &  2.50\%  &  2.59\% \\[7pt]
    \multirow{2}{*}{13\,TeV}  &  ATLAS     &  0.36\%  &  2.29\%  &  2.32\% \\
                              &  CMS       &  0.36\%  &  2.31\%  &  2.34\% \\
    \end{tabular} 
    \end{table*} 

\end{enumerate}

The uncertainties on the predicted cross sections (due to the PDF, the
top-quark mass and the renormalisation and factorisation scale) 
are generally strongly correlated. 
%
The combination result strongly depends on the assumed correlation
structure of these theoretical uncertainties if included in the
combination, which is usually not known precisely in particular for
the scale uncertainty.
%
We therefore adopt
a different procedure: The individual results are
simultaneously shifted up and down by their respective total theory uncertainties, and the
combination is re-evaluated.
% 
The difference between the upper and
lower bounds and the original combination is taken to be the
(asymmetric) theoretical uncertainty.

The impact of the alternative procedure of including also the theory
uncertainties within a single combination is discussed in
Appendix~\ref{sec:appendix}.

%----------------------------------------------------------------------
\subsubsection{Combining correlated measurements: Likelihood-based approach}

In order to combine the individual results, we opted for a
likelihood-based approach~\cite{Cowan:2010js}.\footnote{As a cross-check, we also
  used the \textit{Best Linear Unbiased Estimate} procedure
  (BLUE)~\cite{BLUE1,Valassi:2003mu}. This is only suitable for symmetric
  errors and in that case we found essentially identical results.}
%
In this
approach a global likelihood function is constructed from the
probability distribution functions of individual determinations.  
%
Let us suppose we have $n_m$ measurements of the top cross section and
associated determinations of $\as$. 
%
For each determination $i$, $\as{}_{,i}$, we have $n_u$ uncorrelated
error components, each specific to that determination.
%
The magnitude of the $k^\text{th}$ uncorrelated error for
determination $i$ is labelled $\Delta^k_i$.
%
We additionally have $n_c$ error components that are correlated across
all determinations.
%
For each of the correlated components, $j$, we introduce a nuisance
parameter $\theta_j$ that is common across all measurements.
%
Its impact on measurement $i$ is governed by a coefficient
$\delta_i^j$. 
%
The full set of $\theta_j$ will be denoted $\bm\theta$.

The likelihood will be composed of a product of probability
distribution functions (pdf)\footnote{
\textit{pdf}, for probability density function, is not to be confused with \textit{PDF}, for parton distribution function.
}.
% \siggi{
% We should not use the term ``pdf'' here as that is being
% used for parton density functions already!
% Say ``gdf'' here instead?
% }
%
For each nuisance parameter we will have one pdf, a Gaussian
distribution with a standard deviation of one:
% 
\begin{linenomath*}
\begin{equation}
    \text{pdf}_{\theta_j} = 
    \frac{1}{\sqrt{2\pi}} e^{-\theta_j^2/2}
    %\text{Gauss}( \, \theta_j,\, 0,\, 1 \,)
    \,.
\end{equation}
\end{linenomath*}
%
There will also be a pdf for each combination of measurement $i$ and
associated uncorrelated error $\Delta^k_i$. 
%
It is given by
% 
\begin{linenomath*}
\begin{equation}
\text{pdf}_{i,\,k}(\as, \bm{\theta}) =
    \frac{1}{\sqrt{2\pi} \Delta_i^k}
    \, \exp{ \displaystyle\left[
    -\frac{
        (\as {}_{,i} \, + \sum_j \theta_j \cdot \delta_i^j - \as)^2
        }{
        2(\Delta_i^k)^2
        }
    \right] }
    \;.
\end{equation}
\end{linenomath*}
%
To address the issue of errors that are not symmetric, we adopt the
following prescription for the $\Delta_i^k$ and $\delta_i^j$:
% 
\begin{linenomath*}
\begin{equation}
\Delta_i^k =
    \left\{
    \begin{array}{ll}
        \displaystyle
        \Delta_i^{k,\,-}
        & \quad \text{if} \;\; \as \le \as {}_{,i}
        \\[20pt]
        \displaystyle
        \Delta_i^{k,\,+}
        & \quad \text{if} \;\; \as > \as {}_{,i}
    \end{array}
    \right. 
    \;,
\end{equation}
\end{linenomath*}
% 
\begin{linenomath*}
\begin{equation}
\delta_i^j =
    \left\{
    \begin{array}{ll}
        \displaystyle
        \delta_i^{j,\,-}
        & \quad \text{if} \;\; \as \le \as {}_{,i}
        \\[20pt]
        \displaystyle
        \delta_i^{j,\,+}
        & \quad \text{if} \;\; \as > \as {}_{,i}
    \end{array}
    \right. 
    \;.
\end{equation}
\end{linenomath*}
% 
An overview of the values used for $\delta_i^{j,\,\pm}$ and $\Delta_i^{k,\,\pm}$
is given in Appendix~\ref{sec:appendix2}.
% 
The probability distribution function of determination $i$ including
all uncorrelated uncertainties is then constructed by convolution:
% 
\begin{linenomath*}
\begin{equation}
\text{pdf}_{ \as {}_{,i} }(\as, \bm{\theta}) =
    \text{pdf}_{i,\,1}(\as, \bm{\theta}) \otimes \text{pdf}_{i,\,2}(\as, \bm{\theta})
    \otimes \cdots
    % \otimes \text{pdf}_{i,\,j}(\as, \bm{\theta})
    \otimes \text{pdf}_{i,\,n_u}(\as, \bm{\theta})
\end{equation}
\end{linenomath*}
% 
where the convolution is performed such that the probability distribution
functions are centred around $\as{}_{,i}$.
% 
% where the convolution is defined in analogy with
% Eq.~(\ref{eq:convolution-operator}), replacing $\sigma^\text{pred}$
% with $\as{}_{,i}$.
% 
The global likelihood function $L( \as , \bm{\theta} )$ is constructed
by multiplication of the probability distribution functions of the
determinations and the nuisance parameters:
% 
\begin{linenomath*}
\begin{equation}
L( \as , \bm{\theta} ) = 
    \prod_{i=1}^{n_m} \text{pdf}_{ \as {}_{,i} }(\as, \bm{\theta})
    \, \times \,
    \prod_{j=1}^{n_c} \text{pdf}_{\theta_j}
    \,.
    \label{eq:globallikelihoodfunction}
\end{equation}
\end{linenomath*}
% 
In order to complete the formalism of a statistical test the test
statistic $q$ is introduced:
%
\begin{linenomath*}
\begin{equation}
q( \as ) = -2 \log \frac{ L( \as ,\, \hat{\bm{\theta}}' ) }{ L( \hat{\alpha}_s ,\, \hat{\bm{\theta}} ) }
\;.
\end{equation}
\end{linenomath*}
% 
Here $L$ is maximized for variables that carry a hat and in general
$\hat{\bm{\theta}}'$ will take on different values from
$\hat{\bm{\theta}}$.
%
The quantity $L( \hat{\alpha}_s ,\, \hat{\bm{\theta}} )$ is therefore
the global maximum likelihood, and the ratio cannot be larger than
one. 
%
The normalisation is such
that $q$ can be treated as $\chi^2$-distributed with one degree of
freedom.

The test statistic $q$ is scanned over a range of $\as$ values. The
minimum of the scan, by construction at $q=0$, is the maximum
likelihood value for $\as$, and the $1\sigma$ confidence interval is
extracted from the interval between the intersection points of the
scan with $q=1$. Any skewness of the parabola of the scan is due to
the inclusion of asymmetric uncertainties.
% Fig.~\ref{fig:CT14combination} and 
Figure~\ref{fig:ScanResults} shows the scan and the corresponding
combination results for each of the PDF sets. 

