\subsection{Determination of $\as$ from $\ttbar$ cross section measurements}

\subsubsection{Theory prediction for the top pair production cross section $\stt$}
\label{sec:theory-predictions}

Theory predictions for the dependence of $\stt$ on $\as$ are
calculated using the program
\texttt{top++2.0}~\cite{Czakon:2011xx}. 
%
It provides the computation of the total cross section up to
NNLO~\cite{Czakon:2013goa}, with possible inclusion of soft-gluon
resummation at next-to-next-to-leading logarithmic order (NNLL), as
described in Refs.~\cite{Beneke:2009rj,Czakon:2009zw}.

The predicted cross section is evaluated setting both the
renormalisation scale $\mu_R$ and factorisation scale $\mu_F$ equal to
the top-quark pole mass. 
%
The theoretical uncertainty associated with missing higher-order
contributions is evaluated by independently varying $\mu_R$ and $\mu_F$
up and down by a factor of 2, under the constraint that
$\frac{1}{2} \leq \mu_R / \mu_F \leq 2$.
%
The scale uncertainties are modelled as corresponding to a $68\%$
confidence interval with a Gaussian-shaped uncertainty profile.
%
This choice is more conservative than the (flat) $100\%$ confidence
interval that is sometimes taken for scale variations and used,
notably, in Ref.~\cite{Chatrchyan:2013haa}. 
%
The latter choice leads to a scale
uncertainty contribution that is smaller by a factor $\sqrt{3}$ (the
ratio of the standard deviations of the two uncertainty profiles).
%
Note that a $100\%$ confidence
interval for scale uncertainties is known to be inconsistent with the
observation that a significant fraction of NNLO calculations is
outside the scale uncertainty interval of the corresponding NLO
calculation.\footnote{As discussed in \cite{Bagnaschi:2014wea} and
  also \cite{GavinLHCPtalk}. 
  % 
  Note that the experience with NLO scale uncertainties may not apply
  to NNLO scale uncertainties.
  %
  In particular, for the two cases of hadron-collider calculations
  available at N3LO accuracy, Higgs production in the
  gluon-fusion~\cite{Anastasiou:2016cez} and
  vector-boson-fusion~\cite{Dreyer:2016oyx} channels, while the
  central NNLO results are outside the NLO scale uncertainty bands, the N3LO
  results are well within the corresponding NNLO bands. }

A further choice that needs to be made is whether to include the NNLL
threshold resummation for the cross section.
%
This is a procedure that resums terms whose leading-logarithmic (LL)
structure is $(\as \ln^2 N)^n$, where $N \sim d\ln\sigma_{\ttbar}/d\ln s$ and
$s$ is the squared centre-of-mass energy.
%
When $m_{\ttbar}^2/s$ approaches one, i.e.\ when one approaches the
threshold for $t\bar t$ production, $N$ is proportional to
$1/(1-m_{\ttbar}^2/s$) and the threshold resummation is a necessity.
%
However, at the LHC and even at the Tevatron, top-pair production is
far from threshold and $N$ is not especially large: for the dominant
gluon-gluon production channel at LHC, $N \simeq 1.4$ for
$m_{t\bar t} = 2 \mt$ and $\sqrt{s} = 7\TeV$; while for the dominant
$q\bar q$ production channel at the Tevatron, $N \simeq 1.8$.
%
Accordingly, there is debate within the community as to
whether threshold resummation is called for.
%
On one hand, one may argue that it brings terms that have a certain
physical meaning.
%
On the other, one may argue that there is no reason why the terms
brought by threshold resummation should dominate over other, neglected
terms, and therefore it is more consistent to include just the
fixed-order contributions, which are known exactly.
%
We will take an agnostic approach to this question, carry out fits
with and without NNLL resummation, and then average both the central
values and the uncertainties in the two cases in order to obtain our
final result.

The theory prediction for $\stt$ also depends on a choice of PDF set.
%
Since that choice needs to be related to the data that we fit, we
postpone our discussion of the PDF choice to
section~\ref{sec:pdf-choice}.


%----------------------------------------------------------------------
\subsubsection{Measurements of the top pair production cross section}
\label{sec:data}

Our $\as$ determination is performed using seven $\stt$ inputs, listed
in Table~\ref{tab:includedmeasurements}.
%
The six measurements at the LHC include three updated measurements
by the CMS Collaboration at centre-of-mass energies of 7\TeV,
8\TeV~\cite{Khachatryan:2016mqs} and 13\TeV~\cite{Khachatryan:2015uqb}.  
%
These measurements were performed in the $e \mu$ decay channel,%
\footnote{The $\stt$ measurement by CMS at 13\TeV using events with
  one lepton and at least one jet in the final
  state~\cite{Sirunyan:2017uhy} has a slightly better precision
  than the CMS result used in our analysis. However, the effect on the
  final result is marginal, and using measurements from the same decay
  channel yields a clearer correlation structure for the combination.
}
%
where the $W$-bosons from the top quark decays each themselves decay
into a charged lepton and a neutrino, one of the $W$ decays producing
an electron, the other producing a muon.
% 
The measurements are based on data collected in the years of 2011, 2012 and 2015
respectively, with integrated luminosities of 5.0\fbinv,
19.7\fbinv, and 2.2\fbinv.
%
From the ATLAS Collaboration, three similar
measurements performed in the
$e\mu$ decay channel are included, based on datasets with
integrated luminosities of 4.6\fbinv, 20.3\fbinv and
3.2\fbinv for the 7\TeV, 8\TeV~\cite{Aad:2014kva} and
13\TeV~\cite{Aaboud:2016pbd} centre-of-mass energies
respectively.
%
A seventh input from
the Tevatron collider~\cite{Aaltonen:2013wca} at a centre-of-mass
energy of 1.96\TeV is included, which
comprises a combination of measurements performed in multiple decay
channels from both the CDF Collaboration and the D0 Collaboration.  

\newcommand{\xstableWidth}{2cm}
\begin{table*}[ht]
    \centering
    \setlength\tabcolsep{0pt}
    \renewcommand{\arraystretch}{1.2}
    % 
    \topcaption{
        Cross sections and experimental uncertainties for the
        $\stt$ inputs that we
        use~\cite{Aaltonen:2013wca,Khachatryan:2016mqs,Khachatryan:2015uqb,Aad:2014kva,Aaboud:2016pbd}.
        %
        The LHC beam energy uncertainties quoted in these references have been scaled down by a factor $6.6$
        in light of the recent beam-energy calibration~\cite{Todesco:2017nnk}, which has a $0.1\%$
        uncertainty and coincides
        with the nominal energy within uncertainties. 
        %
        The original beam-energy-induced uncertainties corresponded to
        $0.66\%$~\cite{Wenninger:1546734}.
        %
        The Tevatron beam energy uncertainty is sufficiently
        small (cf.\ Ref.~\cite{Johnson:1988bx}) that no beam energy uncertainty is
        quoted by CDF and D0 in the $t\bar t$ cross section measurements.
        %
        The cross section and uncertainties listed here are adjusted to the
        top mass corresponding to the
        latest world average value computed by the Particle Data Group~\cite{pdg},
        $\mt = 173.2 \pm 0.51 \pm 0.71 \GeV$.
        %
        The ``Exp.\ $\mt \text{ unc.}$''
        column corresponds to the $\delta \mt$ 
        uncertainty discussed in section~\ref{sec:top-mass-dependence},
        signed such that the upper (lower) uncertainty corresponds to an
        increase (decrease) in $\mt$.
        } 
    \label{tab:includedmeasurements}    
    % 
    \begin{tabular}{l c c c c c c }
    \vbox to 10pt {} &
    \cell{1.2cm}{$\stt$ [pb]} &
    \cell{\xstableWidth}{Statistical unc. [\%]} &
    \cell{\xstableWidth}{Systematic unc. [\%]} &
    \cell{\xstableWidth}{Luminosity unc. [\%]} &
    \cell{\xstableWidth}{E$_{\text{beam}}$ unc. [\%]} &
    \cell{\xstableWidth}{Exp. $\mt$ unc. [\%]} \\
    \midrule
    ATLAS (7\,TeV)~\cite{Aad:2014kva}           & $182.5  $ & $1.7 \%$  & $2.3 \%$  & $2.0 \%$  & $0.3 \%$  & ${}^{-0.2\%}_{+0.2\%}$         \\
    ATLAS (8\,TeV)~\cite{Aad:2014kva}           & $242.4  $ & $0.7 \%$  & $2.3 \%$  & $2.1 \%$  & $0.3 \%$  & ${}^{-0.2\%}_{+0.2\%}$         \\
    ATLAS (13\,TeV)~\cite{Aaboud:2016pbd}          & $816.3  $ & $1.0 \%$  & $3.3 \%$  & $2.3 \%$  & $0.2 \%$  & ${}^{-0.3\%}_{+0.3\%}$         \\
    CMS   (7\,TeV)~\cite{Khachatryan:2016mqs}             & $173.4  $ & $1.2 \%$  & $2.5 \%$  & $2.2 \%$  & $0.3 \%$  & ${}^{-0.2\%}_{+0.2\%}$         \\
    CMS   (8\,TeV)~\cite{Khachatryan:2016mqs}             & $244.1  $ & $0.6 \%$  & $2.4 \%$  & $2.6 \%$  & $0.3 \%$  & ${}^{-0.4\%}_{+0.4\%}$         \\
    CMS   (13\,TeV)~\cite{Khachatryan:2015uqb}            & $809.8  $ & $1.1 \%$  & $4.7 \%$  & $2.3 \%$  & $0.2 \%$  & ${}^{-0.8\%}_{+0.8\%}$         \\
    TEV   (1.96\,TeV)~\cite{Aaltonen:2013wca}  & $7.52   $ & $2.7 \%$  & $3.9 \%$  & $2.8 \%$  & $0.0 \%$  & ${}^{-1.1\%}_{+1.4\%}$         \\
    \end{tabular}
\end{table*}


%----------------------------------------------------------------------
\subsubsection{Choice of PDF}
\label{sec:pdf-choice}

Several considerations arise in our choice of PDF. 
%
Firstly, we restrict our attention to recent global fits that are
available through the LHAPDF interface~\cite{Buckley:2014ana}.
%
Secondly, we require that the PDFs should be available for at least
three $\as$ values, so that we can correctly determine the $\as$
dependence of the cross section in the context of that PDF.
%
These two conditions limit us to the CT14~\cite{Dulat:2015mca},
MMHT2014~\cite{Harland-Lang:2014zoa} and the NNPDF3.0~\cite{Ball:2014uwa} series.
%
Thirdly, we impose a requirement that the PDF should not have included
$\stt$ data in its fitting procedure.
%
% As should be obvious qualitatively, and as we will discuss
% quantitatively elsewhere~\cite{InPrep}, using a PDF with top-data
% included would bias our fits.
% 
Qualitatively, it can be expected that using a PDF with top-data included would bias our fits.


\newcommand{\yes}{\checkmark}  % or cmark?
\newcommand{\no}{$-$}
\begin{table}
  \centering
  \setlength\tabcolsep{5pt}
  % 
  \topcaption{%
    Top pair cross section data included in a selection of recent
    PDF fits. A ``\yes'' (``\no'') indicates that the corresponding $t\bar t$
    cross section measurement is (is not)
    included in the PDF fit.
    %
    The specific sets of $7$ and $8\TeV$ ATLAS and CMS data used in
    the fits do not
    always coincide 
    with those that we list in Table~\ref{tab:includedmeasurements}.
    %
    All the PDFs shown here predate the $13\TeV$ measurements.
    }
    \label{tab:pdf-fit-ttbar-choices}
  % 
  \begin{tabular}{lccccc}
    & Tevatron        
    & \cell{1.8cm}{ATLAS (7\TeV)}
    & \cell{1.8cm}{ATLAS (8\TeV)}
    & \cell{1.8cm}{CMS (7\TeV)}
    & \cell{1.8cm}{CMS (8\TeV)}
    \\\midrule
    %%                               TEV    A7     A8     C7     C8
    CT14~\cite{Dulat:2015mca}              & \no  & \no  & \no  & \no  & \no  \\
    MMHT2014~\cite{Harland-Lang:2014zoa}      & \yes & \yes & \no  & \yes & \yes \\
    NNPDF30~\cite{Ball:2014uwa}        & \no  & \yes & \yes & \yes & \yes \\
    NNPDF30\_noLHC~\cite{Ball:2014uwa} & \no  & \no  & \no  & \no  & \no  \\
  \end{tabular}
\end{table}

Table~\ref{tab:pdf-fit-ttbar-choices} summarises what data has been
included in each of these PDF sets, including both the default NNPDF30
set and NNPDF30\_nolhc, obtained without LHC data.
%
One sees that the two options that are available to us are CT14 and
NNPDF30\_nolhc.\footnote{As this article was being completed the
  NNPDF31 series~\cite{Ball:2017nwa} of PDF sets became available. 
  %
  It includes a set fitted without top data, however only
  for a single value of the strong coupling, and accordingly is not
  suitable for use in a strong coupling determination.}


\newcommand{\predxstableWidth}{2.2cm}
\begin{table*}[ht]
    \centering
    \renewcommand{\arraystretch}{1.3}
    \setlength\tabcolsep{3pt}
    % 
    \topcaption{%
      Predicted cross sections and uncertainties for the
      PDF sets that we use~\cite{Dulat:2015mca,Ball:2014uwa}, as determined with the
      \texttt{Top++} program~\cite{Czakon:2011xx} at a reference value
      of $\as^\rf = 0.118$.
      %
      The results are for $\mt = 173.2\GeV$ and the ``$\mt \text{ unc.}$''
      column corresponds to the $\delta \mt$ 
      uncertainty discussed in section~\ref{sec:top-mass-dependence},
      signed such that the upper (lower) uncertainty corresponds to an
      increase (decrease) in $\mt$.
      } 
    \label{tab:includedpredictions}
    % 
    \begin{tabular}{l c c c c c }
    &
    \cell{1.8cm}{$\stt^{\text{pred}}(\as^{\text{ref}})$ [pb]} &
    \cell{1.8cm}{PDF unc. [\%]} &
    \cell{1.8cm}{Scale unc. [\%]} &
    \cell{1.8cm}{$\mt$ unc. [\%]} & 
    $\displaystyle
        \frac{ \text{d}\ln{\stt(\as^{\text{ref}})} }{ \text{d}\ln\as }
        $ \\
    \midrule 
    % File generated on 17-08-15 15:20:14 by the script theoryCentersAndUncertainties.py
% Current git commit: 8ae1f87 Scripts for tables 1 and 3, with the mtop sign now flipped
\textbf{CT14 (NNLO)} & & & & & \\
LHC ($7$\,TeV) &              $172.7$  & ${}_{-3.8\%}^{+4.5\%}$ & ${}_{-6.5\%}^{+4.1\%}$ & ${}^{-2.6\%}_{+2.7\%}$ & $2.486$ \\
LHC ($8$\,TeV) &              $246.7$  & ${}_{-3.5\%}^{+4.0\%}$ & ${}_{-6.3\%}^{+3.9\%}$ & ${}^{-2.5\%}_{+2.6\%}$ & $2.404$ \\
LHC ($13$\,TeV) &             $807.3$  & ${}_{-2.7\%}^{+2.6\%}$ & ${}_{-5.6\%}^{+3.5\%}$ & ${}^{-2.3\%}_{+2.4\%}$ & $2.133$ \\
Tevatron ($1.96$\,TeV) &      $7.3$    & ${}_{-2.2\%}^{+3.4\%}$ & ${}_{-5.5\%}^{+3.8\%}$ & ${}^{-2.7\%}_{+2.8\%}$ & $1.757$ \\
\midrule\multicolumn{6}{l}{\textbf{NNPDF30\_nolhc (NNLO)}} \\
LHC ($7$\,TeV) &              $174.8$  & ${}_{-5.0\%}^{+5.0\%}$ & ${}_{-6.5\%}^{+4.1\%}$ & ${}^{-2.6\%}_{+2.7\%}$ & $2.247$ \\
LHC ($8$\,TeV) &              $249.7$  & ${}_{-4.4\%}^{+4.4\%}$ & ${}_{-6.3\%}^{+3.9\%}$ & ${}^{-2.5\%}_{+2.6\%}$ & $2.099$ \\
LHC ($13$\,TeV) &             $816.2$  & ${}_{-2.9\%}^{+2.9\%}$ & ${}_{-5.6\%}^{+3.5\%}$ & ${}^{-2.3\%}_{+2.4\%}$ & $1.681$ \\
Tevatron ($1.96$\,TeV) &      $7.2$    & ${}_{-3.1\%}^{+3.5\%}$ & ${}_{-5.5\%}^{+3.8\%}$ & ${}^{-2.7\%}_{+2.8\%}$ & $2.396$ \\
\midrule\multicolumn{6}{l}{\textbf{CT14 (NNLO+NNLL)}} \\
LHC ($7$\,TeV) &              $177.9$  & ${}_{-3.7\%}^{+4.4\%}$ & ${}_{-3.5\%}^{+2.6\%}$ & ${}^{-2.6\%}_{+2.7\%}$ & $2.545$ \\
LHC ($8$\,TeV) &              $253.6$  & ${}_{-3.4\%}^{+3.9\%}$ & ${}_{-3.5\%}^{+2.6\%}$ & ${}^{-2.5\%}_{+2.6\%}$ & $2.459$ \\
LHC ($13$\,TeV) &             $825.9$  & ${}_{-2.7\%}^{+2.6\%}$ & ${}_{-3.6\%}^{+2.4\%}$ & ${}^{-2.3\%}_{+2.4\%}$ & $2.178$ \\
Tevatron ($1.96$\,TeV) &      $7.4$    & ${}_{-2.2\%}^{+3.5\%}$ & ${}_{-2.9\%}^{+1.6\%}$ & ${}^{-2.7\%}_{+2.8\%}$ & $1.842$ \\
\midrule\multicolumn{6}{l}{\textbf{NNPDF30\_nolhc (NNLO+NNLL)}} \\
LHC ($7$\,TeV) &              $180.1$  & ${}_{-5.0\%}^{+4.9\%}$ & ${}_{-3.5\%}^{+2.6\%}$ & ${}^{-2.6\%}_{+2.7\%}$ & $2.296$ \\
LHC ($8$\,TeV) &              $256.7$  & ${}_{-4.4\%}^{+4.3\%}$ & ${}_{-3.5\%}^{+2.6\%}$ & ${}^{-2.5\%}_{+2.6\%}$ & $2.147$ \\
LHC ($13$\,TeV) &             $835.0$  & ${}_{-2.8\%}^{+2.8\%}$ & ${}_{-3.6\%}^{+2.4\%}$ & ${}^{-2.3\%}_{+2.4\%}$ & $1.722$ \\
Tevatron ($1.96$\,TeV) &      $7.3$    & ${}_{-3.2\%}^{+3.6\%}$ & ${}_{-2.9\%}^{+1.5\%}$ & ${}^{-2.7\%}_{+2.8\%}$ & $2.476$ \\
    \end{tabular}
    \end{table*}


\newcommand{\FitFigureWidth}{0.44}  
\begin{figure}[t]
\centering
\begin{tabular}{ccc}
\includegraphics[width=\FitFigureWidth\linewidth]{img/alphas/fits_ATLAS7000.pdf}
&
\includegraphics[width=\FitFigureWidth\linewidth]{img/alphas/fits_ATLAS8000.pdf}
\\[-8pt]
(a) & (b) \\[-3pt]
%
\includegraphics[width=\FitFigureWidth\linewidth]{img/alphas/fits_ATLAS13000.pdf}
&
\includegraphics[width=\FitFigureWidth\linewidth]{img/alphas/fits_TEV1960.pdf}
\\[-8pt]
(c) & (d) \\[6pt]
\end{tabular}
\vspace{-0.3cm}
\caption{
  Predicted cross section as a function of $\as$. The points are the cross
  sections calculated using the \texttt{Top++} program~\cite{Czakon:2011xx}, and
  the line is our polynomial fit.
  %
  The plot also includes horizontal lines corresponding to the central
  values of the measured cross sections, adjusted to correspond to the
  same top mass as the theory cross sections ($\mt = \mt^\text{ref} =
  173.2\GeV$), cf.~Section~\ref{sec:top-mass-dependence}.
}
\label{fig:FitsToPrediction}
\end{figure}
% 

We use PDF uncertainties calculated at the 68\% confidence level,
following the error propagation prescription from the individual PDF
groups.
%
The uncertainties from the CT14 PDF set, which are provided at a 90\%
confidence level by default, are scaled by a factor of
$1/(\sqrt2 \, \text{erf}^{-1}( 0.90) )\simeq 0.608 $.

The predicted cross sections for both PDF sets, with NNLO and
NNLO+NNLL calculations, are listed in
Table~\ref{tab:includedpredictions}.
%
The cross sections are $1{-}3\%$ higher when including NNLL
contributions.
%
The scale uncertainties are in the $4{-}6\%$ range for the NNLO
results and get reduced by between one third and one half when
including NNLL terms.
%
At LHC energies, the cross sections with NNPDF30\_nolhc are about $1\%$ larger
than those with CT14, however the opposite pattern is seen at
Tevatron. 
%
Finally, the PDF uncertainties are somewhat larger with NNPDF\_nolhc than
with CT14.

To understand the final errors on the $\as$ determination it is
important also to examine how the predicted cross sections depend on
$\as$, a result of the $\as$ dependence both of the hard cross section
and of the PDFs.
%
This is shown in Fig.~\ref{fig:FitsToPrediction}: points correspond
to the values of $\as$ for which the given PDF is available, and lines
correspond to a fit for $\ln \stt$ using a polynomial of $\ln \as$. 
%
We use polynomials of degree $3$ and $1$ respectively for the CT14 and
NNPDF30\_nolhc PDFs, chosen based on the available number of $\as$ points and
requirements of stability of the extrapolation beyond the available
$\as$ points.
%
A steeper slope of the $\as$ dependence (also quoted at $\as=0.118$ in
the last column of Table~\ref{tab:includedpredictions})
leads to a smaller final error on $\as$ for any given source of
uncertainty on $\stt$. 
%
For LHC energies, CT14 is generally steeper, while at the Tevatron
it is NNPDF\_nolhc that is steeper.
%
Note also that CT14 curves have substantial curvature, and this will
induce asymmetric uncertainties for $\as$, even in the case of
uncertainties on the cross section that are symmetric.



%----------------------------------------------------------------------
\subsubsection{Top-mass dependence}
\label{sec:top-mass-dependence}

The top-quark pole mass is taken to be $173.2 \pm 0.87\,$GeV, which
is consistent with the world average value computed by the Particle Data
Group~\cite{pdg}. 
% 
% 
The experimentally measured cross section, $\stt^{\text{exp}} (\mt)$,
depends on $\mt$ through the acceptance corrections, whose
parametrization is given together with the individual
measurements.
% 
The uncertainty on the experimentally measured cross section due to the top-quark pole mass is given in Tab.~\ref{tab:includedmeasurements}, where the uncertainty was calculated by shifting the top mass up and down by its uncertainty.
% 
An increase in the top mass leads to a decrease in the measured total
cross section.
%
This is because the experiments effectively measure a fiducial cross
section (which is independent of $\mt$) and then extrapolate it to a
total cross section by dividing by the acceptance for the fiducial
cross section.
%
For larger values of $\mt$ the acceptance is larger, since decay
products are more likely to pass transverse momentum cuts, and so the resulting
total cross section is lower.
% 
The theoretically predicted cross section,
$\stt^{\text{pred}} ( \mt )$, also depends on $\mt$, because of the
structure of the underlying hard cross section and the $x$-dependence
of the PDFs, cf.\ Tab.~\ref{tab:includedpredictions}.
%
It too decreases for an increase in the cross section, and this
effect is larger than for the measured cross section.

To define a single error contribution associated with the
top-mass uncertainty, it is convenient to absorb these different
sources of $\mt$ dependence into an effective predicted cross section,
\begin{linenomath*}
\begin{equation}
  \label{eq:eff-mtop-dep}
\sigma_{\ttbar}^{\text{eff}}(\mt) = 
    \stt^{\text{pred}} (  \mt  ) \cdot 
        \frac{
            \stt^{\text{exp}} (  \mt^\rf  )
            }{
            \stt^{\text{exp}} (  \mt  )
            }\,,
\end{equation}
\end{linenomath*}
where $\mt^\rf = 173.2$ is the central value of the world average top
mass.
%
For $\mt = \mt^\rf$, this effective predicted cross section coincides
with the actual predicted one.

The final uncertainty on the effective predicted cross section
associated with the error of $\Delta \mt = 0.87\GeV$ on the world
average top mass is then given by
\begin{linenomath*}
\begin{equation}
  \label{eq:mtop-uncertainty-impact}
  \sigma_{t\bar{t}}^{\text{eff}}(\mt^\rf \pm \Delta \mt) - 
  \sigma_{t\bar{t}}^{\text{eff}}(\mt^\rf)\,.
\end{equation}
\end{linenomath*}
This can be used in our $\as$ determination in a manner similar to
any of the theoretical and PDF uncertainties on the predicted cross
section.
% 
To a good approximation, the final top-mass uncertainty on the
effective cross section is equal to the difference between the
percentage uncertainties in Tabs.~\ref{tab:includedmeasurements} and
\ref{tab:includedpredictions}.
%

%----------------------------------------------------------------------
\subsubsection{Strong coupling determination procedure}
\label{sec:determination-procedure}

In the determination of $\as$ from $\stt$, the theory prediction is
treated as a Bayesian prior (one prior for any given value of $\as$)
and the experimental result as a likelihood function.  The
multiplication of these is the joint posterior probability function
from which $\as$ and its uncertainties are determined after
marginalisation of $\stt$.  The procedure is mostly analogous to that
used by the CMS Collaboration in Ref.~\cite{Chatrchyan:2013haa}.

The construction of the Bayesian prior from the theory dependence
necessitates a single probability distribution function given all
individual theory uncertainties. 
%
The three theory uncertainties are each interpreted as corresponding
to an asymmetric Gaussian function:
%
\begin{linenomath*}
\begin{equation}
f^{\text{Unc. source}} \; (\stt \,|\, \as) = 
\left\{
    \begin{array}{ll}
        \frac{1}{\sqrt{2\pi} \Delta_-}
        \, \text{e}  \,^{-\frac12 \left(
        \frac{
            \stt - \stt^{\text{pred}}(\as)
            }{
            \Delta_-
            }
        \right)^2 }
        & \quad \text{if} \; \stt \leq \stt^{\text{pred}}
        \\[20pt]
        \frac{1}{\sqrt{2\pi} \Delta_+}
        \, \text{e}  \,^{ -\frac12\left(
        \frac{
            \stt - \stt^{\text{pred}}(\as)
            }{
            \Delta_+
            }        
        \right)^2 }
        & \quad \text{if} \; \stt > \stt^{\text{pred}}
    \end{array}
    \right.
,
\label{eq:asymgauss}
\end{equation}
\end{linenomath*}
%
where $\stt^{\text{pred}}(\as)$ is the predicted central value at a
given value of $\as$, and $\Delta_{+(-)}$ is the positive (negative)
uncertainty from a given theory uncertainty source. This function has
the advantage that the integral normalizes naturally to one, and that
the integral from $(\stt^{\text{pred}}-\Delta_-)$ to
$(\stt^{\text{pred}}+\Delta_+)$ corresponds to a 68\% 
confidence interval.  
% 
On average there is a 20\% difference between $\Delta_+$ and $\Delta_-$,
and up to a difference of about 85\% for the most asymmetric uncertainty.
%
The central value for $\stt$ corresponds to the median of the
distribution.

The combined probability distribution function of the predicted cross
section, $f^{\text{pred}}(\stt\,|\,\as)$, is computed by taking the
numerical convolution of the individual asymmetric Gaussian functions:
% 
\begin{linenomath*}
\begin{equation}
f^{\text{pred}}(\stt\,|\,\as) = 
    f^{\text{PDF}}(\stt\,|\,\as)    \otimes
    f^{\mt}(\stt\,|\,\as)           \otimes
    f^{\text{Scale}}(\stt\,|\,\as)\,,
\end{equation}
\end{linenomath*}
% 
where the convolution is performed such that the probability distribution
functions are centred around $\stt^\text{pred}$.
% % 
%
While the individual uncertainty distributions contain a discontinuity
at $\stt = \stt^{\text{pred}}(\as)$, the convolution is a smooth
function.
%
The dependence on $\as$ of the width of the uncertainty band is neglected.%
% 
\footnote{ With this approach of fixed absolute uncertainties on $\stt$,
    theory uncertainties on $\as$ will turn out relatively smaller for
    determinations with a higher central $\as$ value.
    %
    One concern is that this might affect the relative weights of
    different determinations in the combination that is described
    later in Sect.~\ref{sec:combination}.
    % 
    To address this concern, a cross-check was performed in which the
    individual theory errors from our procedure were scaled relative
    to the default approach by a factor
    $\frac{ \as^\text{determination} }{ \as^\text{ref} }$.
    %
    That is equivalent to taking fixed relative (rather than fixed
    absolute) theory uncertainties on $\stt$.
    %
    With the combination procedure of section~\ref{sec:combination},
    the difference induced by this change was below the per mille
    level.
    %
    For the alternative combination procedure in
    Appendix~\ref{sec:appendix}, the effect is less than half a 
    percent on $\as$, which remains much smaller than the difference
    between the two combination procedures.  }
%
The probability distribution function of the predicted cross section
is multiplied by the probability distribution function of the measured
cross section $f^{\text{exp}}(\stt\,|\,\as)$, yielding the joint
Bayesian posterior in terms of $\as$ and $\stt$. The Bayesian
confidence interval of $\as$ can be computed through marginalisation
of the posterior by integrating over $\stt$:
%
\begin{linenomath*}
\begin{equation}
  \label{eq:likelihood-master-equation}
  L(\as) = \int f^{\text{pred}}(\stt\,|\,\as) 
  \cdot 
  f^{\text{exp}}(\stt\,|\,\as) \; \text{d}\stt\,.
\end{equation}
\end{linenomath*}
%
Here, $f^{\text{exp}}(\stt\,|\,\as)$ is taken to be independent of
$\as$. Technically a small dependence on $\as$ is introduced in
$f^{\text{exp}}(\stt\,|\,\as)$ through the acceptance corrections;
however, in the region of relevance around $\as^{\text{ref}} = 0.118$,
the effect of this on the uncertainty of the cross section is below
the percent level~\cite{Chatrchyan:2013haa}, and can thus be safely
neglected.  
%
The marginalised joint posterior $L(\as)$ can be treated
as a probability distribution function. 
%
The central value for the $\as$ determination is taken to be the
location of the peak of $L(\as)$, and the
%
uncertainty is extracted by computing the 68\% confidence interval
whose left and right bounds are at equal height.\footnote{This is
  somewhat different from the prescription to define an asymmetric
  probability distribution in Equation~(\ref{eq:asymgauss}), but coincides
  with widespread practice in ATLAS and CMS likelihood fits.}
%
The procedure is illustrated in Fig.~\ref{fig:fullproc}, showing the
experimental and theory probability distribution functions and the
unmarginalised posterior (Fig.~\ref{fig:fullproc}(a)) as well as the
marginalised posterior with extracted central value and uncertainties
(Fig.~\ref{fig:fullproc}(b)).


\begin{figure}[htb]
\centering
\begin{tabular}{cc}
\includegraphics[width=0.5\linewidth]{img/alphas/fullproc.pdf}
&
\includegraphics[width=0.5\linewidth]{img/alphas/alphasDistribution.pdf}
\\
(a) & (b)
\end{tabular}
\vspace{-0.3cm}
\caption{
  (a) The central values and 1$\sigma$ deviations for the predicted cross
  section ($f^{\text{pred}}(\stt\,|\,\as)$, blue oblique lines) and
  %
  the experimental cross section ($f^{\text{exp}}(\stt\,|\,\as)$,
  red horizontal lines) 
  %
  and the product of the probability distribution functions (green
  shading).
  The markers on the predicted cross
  section indicate the fit points from \texttt{top++2.0}. (b)
  Marginalisation of the joint posterior with Bayesian confidence
  interval.
  }
\label{fig:fullproc}
\end{figure}


The combination of determinations from different experiments
necessitates a breakdown of the total uncertainty into components that
can be assigned to the individual uncertainty sources.  To this end,
the determination is repeated each time omitting a different
uncertainty source, and the squared difference of the resulting
uncertainty with respect to the total uncertainty is computed.  A
relative contribution to the total uncertainty is then computed per
uncertainty source. 

%----------------------------------------------------------------------
\subsubsection{Individual results for $\as$ per $\sigma_{t\bar t}$ measurement}
\label{sec:results-by-exp}

The results of our $\as$ determination are listed for the CT14nnlo
PDF set in Tables~\ref{tab:determination_NNLO_CT14} and
\ref{tab:determination_NNLO_NNLL_CT14} and for the
NNPDF30\_nolhc PDF set in
Tables~\ref{tab:determination_NNLO_NNPDF30nolhc} and
\ref{tab:determination_NNLO_NNLL_NNPDF30nolhc}.

The individual $\as$ determinations are all compatible with the world
average to within uncertainties.
%
The central values are rather similar with the CT14 and NNPDF sets.
%
The largest individual sources of uncertainty on $\as$ are the PDF
uncertainties and the scale uncertainties.
%
For the LHC determinations, the PDF uncertainties tend to be larger
with NNPDF, in part a consequence of the larger uncertainties in the
cross section in Table~\ref{tab:includedpredictions}. 
%
However the other uncertainties are also larger with NNPDF, because
of its weaker dependence on $\as$.

The NNLO+NNLL determinations all have smaller $\as$ results,
consistent with the larger cross sections in
Table~\ref{tab:includedpredictions}.
%
The scale uncertainties are also noticeably smaller.
%
Other uncertainties are largely unchanged.

A final comment concerns the somewhat larger scale, $\mt$ and PDF
uncertainties with the CT14 PDF for the CMS 7 TeV case as compared to
the ATLAS 7 TeV case, or also ATLAS 8 TeV as compared to ATLAS 7 TeV.
%
In general with the CT14 PDF, a smaller value of $\as$ corresponds to
larger uncertainties, because the $\as$ dependence of the cross
section is weaker for small $\as$ values,
cf.~Fig.~\ref{fig:FitsToPrediction}. 
%
Note however, that the scale and other uncertainties on the cross
section predictions have been evaluated only for the reference value
of $\as=0.118$, and in general the question of how one should
correlate uncertainties with the central value is a delicate
one.\footnote{As an example, imagine that we had used scale
  uncertainties that depended on $\as$: then for an experimental
  measurement with cross section that fluctuates low, one would deduce
  a smaller scale uncertainty than for a cross section that fluctuates
  high; when combining them, depending on the procedure, this might
  then lead to a larger weight for the smaller value of $\as$.}
%
Accordingly one should be wary of reading too much into the variation
of uncertainties with the central $\as$ value.


\newcommand{\ErrTableWidth}{0.8cm}
\begin{table*}[ht] 
\centering
\renewcommand{\arraystretch}{1.4}
% 
\topcaption{
    Results for the strong coupling evaluated at the Z-boson mass scale
    and individual uncertainty contributions. 
    %
    These are based on cross sections calculated at NNLO
    using the CT14nnlo series of PDFs.
    }
\label{tab:determination_NNLO_CT14}
% 
\begin{tabular}{l c c c c c c c c c }
&
\cell{\ErrTableWidth}{Center} & 
\cell{\ErrTableWidth}{Stat.} & 
\cell{\ErrTableWidth}{Syst.} & 
\cell{\ErrTableWidth}{Lumi.} & 
\cell{\ErrTableWidth}{$E_{\text{beam}}$} & 
\cell{\ErrTableWidth}{PDF} & 
\cell{\ErrTableWidth}{Scale} & 
\cell{\ErrTableWidth}{$\mt$} & 
\cell{\ErrTableWidth}{Total} \\ 
\midrule
% File generated on 17-07-10 13:19:16 by the script resultTable.py
% Current git commit: bedf7af Script to generate inputtable tex files for the determination result tables. Now has a creation time included and also the Tevatron is in the table for NNPDF30.
ATLAS (7\,TeV)        & 0.1205 & ${}_{-0.0009}^{+0.0007}$ & ${}_{-0.0012}^{+0.0009}$ & ${}_{-0.0010}^{+0.0008}$ & ${}_{-0.0001}^{+0.0001}$ & ${}_{-0.0021}^{+0.0015}$ & ${}_{-0.0021}^{+0.0021}$ & ${}_{-0.0012}^{+0.0009}$ & ${}_{-0.0036}^{+0.0030}$ \\
ATLAS (8\,TeV)        & 0.1171 & ${}_{-0.0004}^{+0.0003}$ & ${}_{-0.0014}^{+0.0011}$ & ${}_{-0.0013}^{+0.0010}$ & ${}_{-0.0002}^{+0.0001}$ & ${}_{-0.0025}^{+0.0017}$ & ${}_{-0.0026}^{+0.0027}$ & ${}_{-0.0015}^{+0.0011}$ & ${}_{-0.0044}^{+0.0037}$ \\
ATLAS (13\,TeV)       & 0.1187 & ${}_{-0.0006}^{+0.0006}$ & ${}_{-0.0021}^{+0.0017}$ & ${}_{-0.0014}^{+0.0012}$ & ${}_{-0.0001}^{+0.0001}$ & ${}_{-0.0016}^{+0.0014}$ & ${}_{-0.0024}^{+0.0026}$ & ${}_{-0.0013}^{+0.0011}$ & ${}_{-0.0041}^{+0.0038}$ \\
CMS (7\,TeV)          & 0.1182 & ${}_{-0.0007}^{+0.0005}$ & ${}_{-0.0014}^{+0.0010}$ & ${}_{-0.0013}^{+0.0009}$ & ${}_{-0.0002}^{+0.0001}$ & ${}_{-0.0025}^{+0.0017}$ & ${}_{-0.0025}^{+0.0025}$ & ${}_{-0.0014}^{+0.0010}$ & ${}_{-0.0043}^{+0.0035}$ \\
CMS (8\,TeV)          & 0.1175 & ${}_{-0.0004}^{+0.0003}$ & ${}_{-0.0015}^{+0.0011}$ & ${}_{-0.0016}^{+0.0012}$ & ${}_{-0.0001}^{+0.0001}$ & ${}_{-0.0024}^{+0.0017}$ & ${}_{-0.0026}^{+0.0026}$ & ${}_{-0.0014}^{+0.0010}$ & ${}_{-0.0044}^{+0.0037}$ \\
CMS (13\,TeV)         & 0.1183 & ${}_{-0.0007}^{+0.0006}$ & ${}_{-0.0030}^{+0.0025}$ & ${}_{-0.0015}^{+0.0013}$ & ${}_{-0.0001}^{+0.0002}$ & ${}_{-0.0017}^{+0.0014}$ & ${}_{-0.0025}^{+0.0026}$ & ${}_{-0.0010}^{+0.0009}$ & ${}_{-0.0047}^{+0.0042}$ \\
Tevatron (1.96\,TeV)  & 0.1202 & ${}_{-0.0018}^{+0.0013}$ & ${}_{-0.0026}^{+0.0019}$ & ${}_{-0.0019}^{+0.0014}$ & ${}_{-0.0000}^{+0.0000}$ & ${}_{-0.0020}^{+0.0014}$ & ${}_{-0.0027}^{+0.0024}$ & ${}_{-0.0009}^{+0.0006}$ & ${}_{-0.0050}^{+0.0039}$ \\

\end{tabular} 
\end{table*} 


\begin{table*}[ht] 
\centering
\renewcommand{\arraystretch}{1.4}
% 
\topcaption{As in Table.~\ref{tab:determination_NNLO_CT14}, but
  now using NNLO cross sections with the NNPDF30\_nolhc series of PDFs.} 
\label{tab:determination_NNLO_NNPDF30nolhc}
% 
\begin{tabular}{l c c c c c c c c c }
&
\cell{\ErrTableWidth}{Center} & 
\cell{\ErrTableWidth}{Stat.} & 
\cell{\ErrTableWidth}{Syst.} & 
\cell{\ErrTableWidth}{Lumi.} & 
\cell{\ErrTableWidth}{$E_{\text{beam}}$} & 
\cell{\ErrTableWidth}{PDF} & 
\cell{\ErrTableWidth}{Scale} & 
\cell{\ErrTableWidth}{$\mt$} & 
\cell{\ErrTableWidth}{Total} \\ 
\midrule
% File generated on 17-07-10 13:19:16 by the script resultTable.py
% Current git commit: bedf7af Script to generate inputtable tex files for the determination result tables. Now has a creation time included and also the Tevatron is in the table for NNPDF30.
ATLAS (7\,TeV)        & 0.1206 & ${}_{-0.0009}^{+0.0009}$ & ${}_{-0.0013}^{+0.0012}$ & ${}_{-0.0011}^{+0.0010}$ & ${}_{-0.0001}^{+0.0002}$ & ${}_{-0.0027}^{+0.0025}$ & ${}_{-0.0025}^{+0.0029}$ & ${}_{-0.0013}^{+0.0012}$ & ${}_{-0.0043}^{+0.0044}$ \\
ATLAS (8\,TeV)        & 0.1166 & ${}_{-0.0004}^{+0.0004}$ & ${}_{-0.0013}^{+0.0012}$ & ${}_{-0.0012}^{+0.0011}$ & ${}_{-0.0002}^{+0.0001}$ & ${}_{-0.0026}^{+0.0024}$ & ${}_{-0.0026}^{+0.0032}$ & ${}_{-0.0014}^{+0.0013}$ & ${}_{-0.0043}^{+0.0045}$ \\
ATLAS (13\,TeV)       & 0.1183 & ${}_{-0.0007}^{+0.0007}$ & ${}_{-0.0024}^{+0.0022}$ & ${}_{-0.0017}^{+0.0016}$ & ${}_{-0.0001}^{+0.0002}$ & ${}_{-0.0021}^{+0.0020}$ & ${}_{-0.0029}^{+0.0035}$ & ${}_{-0.0015}^{+0.0015}$ & ${}_{-0.0049}^{+0.0051}$ \\
CMS (7\,TeV)          & 0.1179 & ${}_{-0.0007}^{+0.0006}$ & ${}_{-0.0013}^{+0.0013}$ & ${}_{-0.0012}^{+0.0011}$ & ${}_{-0.0001}^{+0.0001}$ & ${}_{-0.0028}^{+0.0025}$ & ${}_{-0.0025}^{+0.0030}$ & ${}_{-0.0013}^{+0.0012}$ & ${}_{-0.0044}^{+0.0045}$ \\
CMS (8\,TeV)          & 0.1170 & ${}_{-0.0003}^{+0.0003}$ & ${}_{-0.0014}^{+0.0013}$ & ${}_{-0.0015}^{+0.0014}$ & ${}_{-0.0002}^{+0.0001}$ & ${}_{-0.0026}^{+0.0024}$ & ${}_{-0.0026}^{+0.0032}$ & ${}_{-0.0013}^{+0.0012}$ & ${}_{-0.0044}^{+0.0046}$ \\
CMS (13\,TeV)         & 0.1178 & ${}_{-0.0008}^{+0.0008}$ & ${}_{-0.0034}^{+0.0032}$ & ${}_{-0.0017}^{+0.0016}$ & ${}_{-0.0002}^{+0.0003}$ & ${}_{-0.0021}^{+0.0020}$ & ${}_{-0.0029}^{+0.0034}$ & ${}_{-0.0011}^{+0.0011}$ & ${}_{-0.0054}^{+0.0055}$ \\
Tevatron (1.96\,TeV)  & 0.1205 & ${}_{-0.0014}^{+0.0013}$ & ${}_{-0.0020}^{+0.0019}$ & ${}_{-0.0014}^{+0.0014}$ & ${}_{-0.0000}^{+0.0000}$ & ${}_{-0.0017}^{+0.0015}$ & ${}_{-0.0021}^{+0.0023}$ & ${}_{-0.0007}^{+0.0007}$ & ${}_{-0.0040}^{+0.0039}$ \\

\end{tabular} 
\end{table*} 


\begin{table*}[ht] 
\centering
\renewcommand{\arraystretch}{1.4}
% 
\topcaption{As in Table.~\ref{tab:determination_NNLO_CT14}, but
  now using NNLO+NNLL cross sections with the CT14nnlo series of PDFs.}
\label{tab:determination_NNLO_NNLL_CT14}
% 
\begin{tabular}{l c c c c c c c c c }
&
\cell{\ErrTableWidth}{Center} & 
\cell{\ErrTableWidth}{Stat.} & 
\cell{\ErrTableWidth}{Syst.} & 
\cell{\ErrTableWidth}{Lumi.} & 
\cell{\ErrTableWidth}{$E_{\text{beam}}$} & 
\cell{\ErrTableWidth}{PDF} & 
\cell{\ErrTableWidth}{Scale} & 
\cell{\ErrTableWidth}{$\mt$} & 
\cell{\ErrTableWidth}{Total} \\ 
\midrule
\input{img/alphas/resultTable_Jul10_NNLO+NNLL-CT14nnlo_asym.tex}
\end{tabular} 
\end{table*} 


\begin{table*}[ht] 
\centering
\renewcommand{\arraystretch}{1.4}
% 
\topcaption{As in Table.~\ref{tab:determination_NNLO_CT14}, but
  now using NNLO+NNLL cross sections with the NNPDF30\_nolhc series of PDFs.} 
\label{tab:determination_NNLO_NNLL_NNPDF30nolhc}
% 
\begin{tabular}{l c c c c c c c c c }
&
\cell{\ErrTableWidth}{Center} & 
\cell{\ErrTableWidth}{Stat.} & 
\cell{\ErrTableWidth}{Syst.} & 
\cell{\ErrTableWidth}{Lumi.} & 
\cell{\ErrTableWidth}{$E_{\text{beam}}$} & 
\cell{\ErrTableWidth}{PDF} & 
\cell{\ErrTableWidth}{Scale} & 
\cell{\ErrTableWidth}{$\mt$} & 
\cell{\ErrTableWidth}{Total} \\ 
\midrule
% File generated on 17-07-10 13:19:16 by the script resultTable.py
% Current git commit: bedf7af Script to generate inputtable tex files for the determination result tables. Now has a creation time included and also the Tevatron is in the table for NNPDF30.
ATLAS (7\,TeV)        & 0.1190 & ${}_{-0.0009}^{+0.0009}$ & ${}_{-0.0012}^{+0.0012}$ & ${}_{-0.0011}^{+0.0010}$ & ${}_{-0.0001}^{+0.0001}$ & ${}_{-0.0026}^{+0.0025}$ & ${}_{-0.0015}^{+0.0016}$ & ${}_{-0.0013}^{+0.0012}$ & ${}_{-0.0037}^{+0.0036}$ \\
ATLAS (8\,TeV)        & 0.1152 & ${}_{-0.0004}^{+0.0004}$ & ${}_{-0.0013}^{+0.0012}$ & ${}_{-0.0012}^{+0.0011}$ & ${}_{-0.0001}^{+0.0001}$ & ${}_{-0.0025}^{+0.0024}$ & ${}_{-0.0017}^{+0.0018}$ & ${}_{-0.0014}^{+0.0013}$ & ${}_{-0.0037}^{+0.0037}$ \\
ATLAS (13\,TeV)       & 0.1168 & ${}_{-0.0007}^{+0.0007}$ & ${}_{-0.0023}^{+0.0022}$ & ${}_{-0.0016}^{+0.0015}$ & ${}_{-0.0002}^{+0.0002}$ & ${}_{-0.0020}^{+0.0019}$ & ${}_{-0.0020}^{+0.0022}$ & ${}_{-0.0015}^{+0.0015}$ & ${}_{-0.0043}^{+0.0043}$ \\
CMS (7\,TeV)          & 0.1163 & ${}_{-0.0006}^{+0.0006}$ & ${}_{-0.0013}^{+0.0012}$ & ${}_{-0.0011}^{+0.0011}$ & ${}_{-0.0001}^{+0.0001}$ & ${}_{-0.0027}^{+0.0026}$ & ${}_{-0.0016}^{+0.0016}$ & ${}_{-0.0013}^{+0.0012}$ & ${}_{-0.0038}^{+0.0037}$ \\
CMS (8\,TeV)          & 0.1155 & ${}_{-0.0003}^{+0.0003}$ & ${}_{-0.0013}^{+0.0013}$ & ${}_{-0.0014}^{+0.0014}$ & ${}_{-0.0001}^{+0.0001}$ & ${}_{-0.0025}^{+0.0024}$ & ${}_{-0.0017}^{+0.0017}$ & ${}_{-0.0013}^{+0.0012}$ & ${}_{-0.0038}^{+0.0037}$ \\
CMS (13\,TeV)         & 0.1163 & ${}_{-0.0007}^{+0.0008}$ & ${}_{-0.0032}^{+0.0031}$ & ${}_{-0.0016}^{+0.0015}$ & ${}_{-0.0002}^{+0.0002}$ & ${}_{-0.0020}^{+0.0019}$ & ${}_{-0.0020}^{+0.0022}$ & ${}_{-0.0011}^{+0.0011}$ & ${}_{-0.0048}^{+0.0047}$ \\
Tevatron (1.96\,TeV)  & 0.1194 & ${}_{-0.0013}^{+0.0013}$ & ${}_{-0.0019}^{+0.0018}$ & ${}_{-0.0014}^{+0.0013}$ & ${}_{-0.0000}^{+0.0000}$ & ${}_{-0.0017}^{+0.0016}$ & ${}_{-0.0010}^{+0.0010}$ & ${}_{-0.0007}^{+0.0007}$ & ${}_{-0.0034}^{+0.0033}$ \\

\end{tabular} 
\end{table*} 
