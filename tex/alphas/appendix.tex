\section{Including strongly correlated uncertainty sources in the combination}
\label{sec:appendix}

Our approach of excluding strongly correlated uncertainties from the combination is generally recommended when using the covariance matrix to fit strongly correlated data~\cite{DAgostini:1993arp}.
% 
To illustrate the effect of strong correlations, the combination is here performed again with the PDF, scale and $\mt$ uncertainties included in the combination one by one.
% 
The PDF and scale uncertainties are considered fully correlated between measurements made with the LHC, and partially correlated between measurements made with the LHC and Tevatron.
% 
In the case of the PDF uncertainties the degree of correlation between LHC and Tevatron measurements was determined using the procedure described in Ref.~\cite{Buckley:2014ana}.
%
The $\mt$ uncertainties are considered fully correlated for all measurements.
% 

Tables~\ref{tab:theoryUncertainties_CT14_NNLO} and \ref{tab:theoryUncertainties_CT14_NNLO_NNLL} show the results for the CT14 PDF set, using NNLO and NNLO+NNLL respectively, and Tables~\ref{tab:theoryUncertainties_NNPDF30_NNLO} and \ref{tab:theoryUncertainties_NNPDF30_NNLO_NNLL} for the NNPDF30\_nolhc PDF set.
% 
As expected, the total uncertainty decreases as more sources are included in the combination.
% 
As the sensitivity to $\as$ is stronger for a larger cross section, determinations that deviate up can have a smaller uncertainty, and therefore obtain a larger weight in the combination.
% 
This is the case for the determination from the ATLAS measurement at 7$\TeV$ when using the CT14 PDF set.
% 
A larger weight may also be obtained for determinations that are more independent with respect to the others. This is primarily the case for the Tevatron determination, for both PDF sets.
% 
These effects are enhanced if the overall correlation is increased by including strongly correlated uncertainty sources, which explains why the combination yields increasing values of $\as$ as more sources are included.
% 
The results found this way are larger than both the straight average and the median of the individual determinations, though the difference is well within one standard deviation. 
% 
Taking them as our final results would imply a high degree of trust in the assumed correlations.
% 
Due to the inherent difficulty of determining correlations, notably as concerns the scale variations, and the importance of the subtle interplay between an individual determination's $\alpha_s$ result and its error, the conservative approach is to exclude the strongly correlated sources from the combination.
%

\begin{table}[ht] 
\renewcommand{\arraystretch}{1.4}
\setlength\tabcolsep{5pt}
\centering
{\scriptsize
    % \begin{center} 
    \hspace*{-0.75cm}%
    \begin{minipage}[t]{0.5\hsize}\centering
    \begin{tabular}{l l l l}
    \toprule
    \lcell{2.4cm}{Uncertainties included in combination} & 
    \lcell{0.7cm}{Center} & 
    \lcell{1.5cm}{Combination uncertainty} & 
    \lcell{1.5cm}{Total uncertainty} \\ 
    \midrule
    % File generated on 17-08-07 13:14:56 by the script HighLevelAnalysis.py
% Current git commit: 2a37c5f Changed table format
-                                         &  $0.1184  $  &  ${}_{-0.0010}^{+0.0009}$  &  ${}_{-0.0038}^{+0.0033}$ \\
PDF                                       &  $0.1191  $  &  ${}_{-0.0018}^{+0.0018}$  &  ${}_{-0.0034}^{+0.0033}$ \\
PDF and $\mt$                             &  $0.1194  $  &  ${}_{-0.0020}^{+0.0020}$  &  ${}_{-0.0032}^{+0.0032}$ \\
PDF, $\mt$ and scale                      &  $0.1207  $  &  ${}_{-0.0029}^{+0.0030}$  &  ${}_{-0.0029}^{+0.0030}$ \\
    \bottomrule
    \end{tabular} 
    % \end{center} 
    \caption{\small
        Combination results including also uncertainties from the PDF, the scale
        and the top mass in the combination.
        % 
        The first row corresponds to our
        approach of excluding correlated uncertainties from the combination.
        % 
        The results listed here are obtained using NNLO cross sections with
        the CT14nnlo series of PDFs.
        }
    \label{tab:theoryUncertainties_CT14_NNLO}
    \end{minipage}%
    % 
    \hspace*{1cm}%
    % 
    % \begin{center} 
    % \hspace*{-0.75cm}%
    \begin{minipage}[t]{0.5\hsize}\centering
    \begin{tabular}{l l l l }
    \toprule
    \lcell{2.4cm}{Uncertainties included in combination} & 
    \lcell{0.7cm}{Center} & 
    \lcell{1.5cm}{Combination uncertainty} & 
    \lcell{1.5cm}{Total uncertainty} \\ 
    \midrule
    % File generated on 17-08-07 13:14:56 by the script HighLevelAnalysis.py
% Current git commit: 2a37c5f Changed table format
-                                         &  $0.1172  $  &  ${}_{-0.0010}^{+0.0010}$  &  ${}_{-0.0033}^{+0.0027}$ \\
PDF                                       &  $0.1180  $  &  ${}_{-0.0020}^{+0.0019}$  &  ${}_{-0.0029}^{+0.0027}$ \\
PDF and $\mt$                             &  $0.1183  $  &  ${}_{-0.0022}^{+0.0022}$  &  ${}_{-0.0027}^{+0.0027}$ \\
PDF, $\mt$ and scale                      &  $0.1188  $  &  ${}_{-0.0025}^{+0.0025}$  &  ${}_{-0.0025}^{+0.0025}$ \\
    \bottomrule
    \end{tabular} 
    % \end{center} 
    \caption{\small
        As in Table~\ref{tab:theoryUncertainties_CT14_NNLO}, but now using NNLO+NNLL
        cross sections with the CT14nnlo series of PDFs.
        }
    \label{tab:theoryUncertainties_CT14_NNLO_NNLL}
    \end{minipage}
    % 
    % NEW LINE
    \\[16pt]
    % 
    \hspace*{-0.75cm}%
    \begin{minipage}[t]{0.5\hsize}\centering
    \begin{tabular}{l l l l }
    \toprule
    \lcell{2.4cm}{Uncertainties included in combination} & 
    \lcell{0.7cm}{Center} & 
    \lcell{1.5cm}{Combination uncertainty} & 
    \lcell{1.5cm}{Total uncertainty} \\ 
    \midrule
    % File generated on 17-08-07 13:14:56 by the script HighLevelAnalysis.py
% Current git commit: 2a37c5f Changed table format
-                                         &  $0.1182  $  &  ${}_{-0.0010}^{+0.0010}$  &  ${}_{-0.0040}^{+0.0042}$ \\
PDF                                       &  $0.1188  $  &  ${}_{-0.0022}^{+0.0023}$  &  ${}_{-0.0037}^{+0.0040}$ \\
PDF and $\mt$                             &  $0.1190  $  &  ${}_{-0.0024}^{+0.0025}$  &  ${}_{-0.0036}^{+0.0040}$ \\
PDF, $\mt$ and scale                      &  $0.1200  $  &  ${}_{-0.0036}^{+0.0035}$  &  ${}_{-0.0036}^{+0.0035}$ \\
    \bottomrule
    \end{tabular} 
    % \end{center} 
    \caption{\small
        As in Table~\ref{tab:theoryUncertainties_CT14_NNLO}, but now using NNLO
        cross sections with the NNPDF30\_nolhc series of PDFs.
        }
    \label{tab:theoryUncertainties_NNPDF30_NNLO}
    \end{minipage}%
    % 
    \hspace*{1cm}%
    % 
    % \begin{center} 
    % \hspace*{-0.75cm}%
    \begin{minipage}[t]{0.5\hsize}\centering
    \begin{tabular}{l l l l }
    \toprule
    \lcell{2.4cm}{Uncertainties included in combination} & 
    \lcell{0.7cm}{Center} & 
    \lcell{1.5cm}{Combination uncertainty} & 
    \lcell{1.5cm}{Total uncertainty} \\ 
    \midrule
    % File generated on 17-08-07 13:14:56 by the script HighLevelAnalysis.py
% Current git commit: 2a37c5f Changed table format
-                                         &  $0.1168  $  &  ${}_{-0.0010}^{+0.0010}$  &  ${}_{-0.0034}^{+0.0033}$ \\
PDF                                       &  $0.1175  $  &  ${}_{-0.0023}^{+0.0023}$  &  ${}_{-0.0031}^{+0.0031}$ \\
PDF and $\mt$                             &  $0.1178  $  &  ${}_{-0.0025}^{+0.0025}$  &  ${}_{-0.0030}^{+0.0030}$ \\
PDF, $\mt$ and scale                      &  $0.1182  $  &  ${}_{-0.0028}^{+0.0028}$  &  ${}_{-0.0028}^{+0.0028}$ \\
    \bottomrule
    \end{tabular} 
    % \end{center} 
    \caption{\small
        As in Table~\ref{tab:theoryUncertainties_CT14_NNLO}, but now using NNLO+NNLL
        cross sections with the NNPDF30\_nolhc series of PDFs.
        }
    \label{tab:theoryUncertainties_NNPDF30_NNLO_NNLL}
    \end{minipage}%
}%
\end{table} 


% ____________________________________________________________________________
\section{Overview of asymmetric uncertainties used in the combination}
\label{sec:appendix2}

Tables~\ref{tab:uncertaintyCoefficients_CT14_NNLO} and \ref{tab:uncertaintyCoefficients_CT14_NNLO_NNLL} show the numerical values for the uncertainty coefficients used in the combination procedure for the CT14 PDF set, using NNLO and NNLO+NNLL cross sections respectively.
% 
Only experimental uncertainties are listed.
% 
Theoretical uncertainties, which are taken into account after the combination procedure, can be found in Tables~\ref{tab:determination_NNLO_CT14}-\ref{tab:determination_NNLO_NNLL_NNPDF30nolhc}.
% 
The correlations for the correlated uncertainties (with a $\delta$ symbol) are described in Section~\ref{sec:correlationcoefficients}.

\begin{table}[ht] 
\renewcommand{\arraystretch}{2.0}
\providecommand{\ErrTableWidth}{1.0cm}
\setlength\tabcolsep{5pt}
\centering
{\footnotesize
    \hspace*{-0.75cm}%
    \begin{minipage}[t]{0.5\hsize}\centering
        \begin{tabular}{l c c c c c }
        \toprule
        % File generated on 17-10-16 17:14:10 by the script printDeltaTable.py
% Current git commit: 38db69e Used wrong bfin files in last commit, these numbers are actually correct
Exp.                        & $\delta^{\text{Syst.}}$     & $\delta^{\text{Lumi.}}$     & $\delta^{\text{E}_{beam}}$  & $\Delta^{\text{Stat.}}$     & $\Delta^{\text{Lumi.}}$     \\
\midrule
\cell{\ErrTableWidth}{ATLAS (13\,TeV)}      & ${}_{-0.0021}^{+0.0017}$    & ${}_{-0.0002}^{+0.0002}$    & ${}_{-0.0001}^{+0.0001}$    & ${}_{-0.0006}^{+0.0006}$    & ${}_{-0.0014}^{+0.0012}$    \\
\cell{\ErrTableWidth}{ATLAS (8\,TeV)}       & ${}_{-0.0014}^{+0.0011}$    & ${}_{-0.0004}^{+0.0003}$    & ${}_{-0.0002}^{+0.0001}$    & ${}_{-0.0004}^{+0.0003}$    & ${}_{-0.0013}^{+0.0009}$    \\
\cell{\ErrTableWidth}{ATLAS (7\,TeV)}       & ${}_{-0.0012}^{+0.0009}$    & ${}_{-0.0003}^{+0.0002}$    & ${}_{-0.0001}^{+0.0001}$    & ${}_{-0.0009}^{+0.0007}$    & ${}_{-0.0010}^{+0.0007}$    \\
\cell{\ErrTableWidth}{CMS (13\,TeV)}        & ${}_{-0.0030}^{+0.0025}$    & ${}_{-0.0002}^{+0.0002}$    & ${}_{-0.0001}^{+0.0002}$    & ${}_{-0.0007}^{+0.0006}$    & ${}_{-0.0015}^{+0.0013}$    \\
\cell{\ErrTableWidth}{CMS (8\,TeV)}         & ${}_{-0.0015}^{+0.0011}$    & ${}_{-0.0004}^{+0.0003}$    & ${}_{-0.0001}^{+0.0001}$    & ${}_{-0.0004}^{+0.0003}$    & ${}_{-0.0016}^{+0.0012}$    \\
\cell{\ErrTableWidth}{CMS (7\,TeV)}         & ${}_{-0.0014}^{+0.0010}$    & ${}_{-0.0003}^{+0.0002}$    & ${}_{-0.0002}^{+0.0001}$    & ${}_{-0.0007}^{+0.0005}$    & ${}_{-0.0013}^{+0.0009}$    \\
\cell{\ErrTableWidth}{Tevatron (1.96\,TeV)} & ${}_{-0.0026}^{+0.0019}$    & -                           & -                           & ${}_{-0.0018}^{+0.0013}$    & ${}_{-0.0019}^{+0.0014}$    \\
        \bottomrule
        \end{tabular} 
        % \end{center} 
        \caption{\small
            Overview of the uncertainty coefficients for the CT14 (NNLO) PDF set.
            % 
            The coefficients with a $\delta$ correspond to the coefficients of the correlated
            uncertainty sources, and those with a $\Delta$ to the uncorrelated uncertainty sources.
            }
        \label{tab:uncertaintyCoefficients_CT14_NNLO}
        \end{minipage}%
    % 
    \hspace*{1cm}%
    % 
    \begin{minipage}[t]{0.5\hsize}\centering
        \begin{tabular}{l c c c c c }
        \toprule
        % File generated on 17-10-16 17:14:10 by the script printDeltaTable.py
% Current git commit: 38db69e Used wrong bfin files in last commit, these numbers are actually correct
Exp.                        & $\delta^{\text{Syst.}}$     & $\delta^{\text{Lumi.}}$     & $\delta^{\text{E}_{beam}}$  & $\Delta^{\text{Stat.}}$     & $\Delta^{\text{Lumi.}}$     \\
\midrule
\cell{\ErrTableWidth}{ATLAS (13\,TeV)}      & ${}_{-0.0020}^{+0.0018}$    & ${}_{-0.0002}^{+0.0002}$    & ${}_{-0.0001}^{+0.0001}$    & ${}_{-0.0006}^{+0.0005}$    & ${}_{-0.0014}^{+0.0012}$    \\
\cell{\ErrTableWidth}{ATLAS (8\,TeV)}       & ${}_{-0.0014}^{+0.0011}$    & ${}_{-0.0004}^{+0.0003}$    & ${}_{-0.0002}^{+0.0001}$    & ${}_{-0.0004}^{+0.0004}$    & ${}_{-0.0013}^{+0.0010}$    \\
\cell{\ErrTableWidth}{ATLAS (7\,TeV)}       & ${}_{-0.0012}^{+0.0010}$    & ${}_{-0.0003}^{+0.0002}$    & ${}_{-0.0001}^{+0.0001}$    & ${}_{-0.0009}^{+0.0007}$    & ${}_{-0.0010}^{+0.0008}$    \\
\cell{\ErrTableWidth}{CMS (13\,TeV)}        & ${}_{-0.0029}^{+0.0025}$    & ${}_{-0.0002}^{+0.0002}$    & ${}_{-0.0002}^{+0.0001}$    & ${}_{-0.0007}^{+0.0006}$    & ${}_{-0.0015}^{+0.0013}$    \\
\cell{\ErrTableWidth}{CMS (8\,TeV)}         & ${}_{-0.0015}^{+0.0012}$    & ${}_{-0.0004}^{+0.0003}$    & ${}_{-0.0002}^{+0.0001}$    & ${}_{-0.0004}^{+0.0003}$    & ${}_{-0.0016}^{+0.0012}$    \\
\cell{\ErrTableWidth}{CMS (7\,TeV)}         & ${}_{-0.0015}^{+0.0011}$    & ${}_{-0.0003}^{+0.0002}$    & ${}_{-0.0002}^{+0.0001}$    & ${}_{-0.0007}^{+0.0006}$    & ${}_{-0.0013}^{+0.0010}$    \\
\cell{\ErrTableWidth}{Tevatron (1.96\,TeV)} & ${}_{-0.0025}^{+0.0021}$    & -                           & -                           & ${}_{-0.0017}^{+0.0014}$    & ${}_{-0.0018}^{+0.0015}$    \\
        \bottomrule
        \end{tabular} 
        \caption{\small
            As in Table~\ref{tab:uncertaintyCoefficients_CT14_NNLO}, but now using
            NNLO+NNLL cross sections with the CT14 PDF set.
            }
        \label{tab:uncertaintyCoefficients_CT14_NNLO_NNLL}
        \end{minipage}%    % 
}%
\end{table} 

